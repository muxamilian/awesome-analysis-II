\documentclass[10pt,a4paper,parskip=half]{scrartcl}
\usepackage[utf8]{inputenc}
\usepackage{amsmath}
\usepackage{amsfonts}
\usepackage{amssymb}
\usepackage{mathpazo}
\usepackage{stmaryrd} % Für den Widerspruchsblitz :D
\usepackage[a4paper,
left=3.0cm, right=3.0cm,
top=2.0cm, bottom=2.0cm]{geometry}
\usepackage{fullpage}
\usepackage[german]{babel}
\usepackage{enumerate}
\setlength{\unitlength}{1cm}
\newcommand{\N}{\mathbb{N}}
\newcommand{\A}{\mathcal{A}}
\newcommand{\R}{\mathbb{R}}
\parindent 0mm
\author{Tom}
\title{Analysis 2 - Hausaufgabe 6}

% new commands for vectors
\newcommand{\vectwo}[2]{\begin{pmatrix}#1\\#2\\\end {pmatrix}}
\newcommand{\vecthree}[3]{\begin{pmatrix}#1\\#2\\#3\\\end {pmatrix}}

\begin{document}
\begin{center}
\textsc{\Large{Analysis 2 - Hausaufgabe 6}} \\
\end{center}
\begin{tabbing}
Tom Nick \hspace{1.4cm}\= 342225\\
Tom Lehmann\> 340621\\
Maximilian Bachl\> 341455
\end{tabbing}
\section*{Aufgabe 1}
Es handelt sich hier um die Suche nach Extrema mit Nebenbedinung, weshalb wir zunächst nach Extrema auf dem Rand des Kreises suchen.\\\\
Die Nebenbedingung lautet: $g(x,y) = x^2 + y^2 - 1 = 0$.
\begin{enumerate}
\item {Singulärer Fall:}
\begin{align*}
\nabla g(\vec x) &= \vec 0 = \begin{pmatrix}2x \\ 2y\end{pmatrix}  \Rightarrow (x,y) = (0,0)
\intertext{Da $g(0,0) = -1 \neq 0$ gibt es hier keinen singulären Fall.}
\end{align*}
\item {$\nabla f( \vec x) = \lambda \cdot \nabla g(\vec x)$}
\begin{align*}
\nabla f(\vec x) &= \begin{pmatrix}6x-2y \\ -2x + 2y \end{pmatrix}= \lambda \begin{pmatrix}2x \\ 2y\end{pmatrix} = \lambda \cdot \nabla g(\vec x)
\intertext{Also:}
6x-2y &= \lambda \cdot 2x \Rightarrow x \text{TODO}\\
-2x + 2y &= \lambda \cdot 2y
\end{align*}
TODO
\end{enumerate}
Wir prüfen zunächst die notwendige Bedingung für kritische Punkte$\nabla f(\vec x_k) = 0$:
\begin{align*}
\nabla f( \vec x) &= \vec 0 = \begin{pmatrix}2\left(x-y\right) + 4x \\ -2\left(x-y\right) \end{pmatrix} = \begin{pmatrix} 6x -2y \\ -2x + 2y\end{pmatrix}
\intertext{Also:}
0 &= 6x-2y	\text{Evtl. Nummerierung hinzufügen}\\
0 &= -2x + 2y \\
\Rightarrow 0 &= 4x \Rightarrow x=0 \Rightarrow y=0
\end{align*}
Wir erhalten deshalb einen kritischen Punkt: $x_{k1} = \begin{pmatrix}0 \\ 0\end{pmatrix}$.\\
Die Hessematrix ist, da es sich bei $f$ um eine zweimal stetig partiell differentierbare Funktion handelt, gemäß dem Satz von Schwarz, invertierbar.
\begin{align*}
f''(\vec x) &= H_f(\vec x) =  \begin{pmatrix}6 & -2 \\ -2 & 2\end{pmatrix}\\
D_1 &= \det (6) = 6 > 0\\
D_2 &= \det \begin{pmatrix} 6 & -2 \\ -2 & 2\end{pmatrix} = 8 > 0
\end{align*}
Damit ist $H_f(\vec x)$ positiv definit, woraus schlusszufolgern ist, dass die Funktion $f$ bei $f(0,0) = 0$ ein lokales Minimum besitzt. Da $f(x,y)$ eine Komposition aus $(x-y)^2 > 0 \; \; \forall x,y \in \mathbb{R}$ und $2x^2 > 0 \; \; \forall x \in \mathbb{R}$ ist, ist $f(0,0) = 0$ sogar ein globales Minimum.
\subsection*{Aufgabe 2}
Um lokale Extrema einer mehrdimensionalen Funktion zu bestimmen, müssen wir 1. die kritischen Punkte finden und 2. diese als Funtkionswerte der Hessematrix übergeben.
\begin{enumerate}
	\item kritische Punkte sind alle Funktionswerte die $\nabla f(x,y,z) = \vec0$ erfüllen.
	\[\nabla f(x,y,z) = 
	\begin{pmatrix}
		4x \\ 
		4z + 2y \\
		4y + 10z	
	\end{pmatrix} \overset{\text{DGL}}{\Rightarrow}
	\begin{pmatrix}
		4 & 0 & 0 \\
		0 & 2 & 4 \\
		0 & 4 & 10 
	\end{pmatrix} \overset{\text{Gauss}}{\Rightarrow}
	\begin{pmatrix}
		1 & 0 & 0 \\
		0 & 1 & 0 \\
		0 & 0 & 1
	\end{pmatrix}\]
	Der Kern der Einheitsmatrix ist der Null-Vektor. Dies ist auch der einzige kritische Punkt demnach.
	\item Hesse-Matrix berechnen und kritische Punkte einfügen:
	\[H_f(x,y,z) =
	\begin{pmatrix}
		4 & 0 & 0 \\
		0 & 2 & 4 \\
		0 & 4 & 10
	\end{pmatrix}
	\] Da die Hesse Matrix konstant ist, müssen wir den Punkt offensichtlich nicht einsetzen. Mit dem Hurwitzkritirium kann nun überprüft, was der kritische Punkt nun ist. 
	\begin{align*}
	 	&\det(4) = 4 \\
	 	&\det \begin{pmatrix}
	 		4 & 0 \\
	 		0 & 2	
	 	\end{pmatrix} = 8 \\
	 	&\det \begin{pmatrix}
	 		4 & 0 & 0 \\
			0 & 2 & 4 \\
			0 & 4 & 10
	 	\end{pmatrix} = 4*2*10 + 0*.. + 0*.. - 0 * .. - 0 *.. - 4 * 2 * 4 = 80 - 32 = 48 
	 \end{align*} Damit ist $H_f(0,0,0)$ positiv definit. Somit ist ein lokales Minimum bei $\vec 0$. Ist es auch ein globales Minimum? 
	$2x^2$, $y^2$ und $5z^2$ werden nie negativ. Also kann nur $4yz$ negativ werden. Da aber $4yz  + y^2 + 5z^2 \ge 0$ für alle $(x,y) \in \mathbb{R}^2$ offensichtlich gilt, kann $f$ keine negativen Funktionswerte haben, somit muss $\vec 0$ ein globales Minimum sein.


\end{enumerate}

\end{document}