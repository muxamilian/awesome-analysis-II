\documentclass[10pt,a4paper,parskip=half]{scrartcl}
\usepackage[utf8]{inputenc}
\usepackage{amsmath}
\usepackage{amsfonts}
\usepackage{amssymb}
\usepackage{mathpazo}
\usepackage{stmaryrd} % Für den Widerspruchsblitz :D
\usepackage[a4paper,
left=3.0cm, right=3.0cm,
top=2.0cm, bottom=2.0cm]{geometry}
\usepackage{fullpage}
\usepackage[german]{babel}
\usepackage{enumerate}
\setlength{\unitlength}{1cm}
\newcommand{\N}{\mathbb{N}}
\newcommand{\A}{\mathcal{A}}
\newcommand{\R}{\mathbb{R}}
\parindent 0mm
\author{Tom}
\title{Analysis 2 - Hausaufgabe 6}

% new commands for vectors
\newcommand{\vectwo}[2]{\begin{pmatrix}#1\\#2\\\end {pmatrix}}
\newcommand{\vecthree}[3]{\begin{pmatrix}#1\\#2\\#3\\\end {pmatrix}}

\begin{document}
\begin{center}
\textsc{\Large{Analysis 2 - Hausaufgabe 6}} \\
\end{center}
\begin{tabbing}
Tom Nick \hspace{1.4cm}\= 342225\\
Tom Lehmann\> 340621\\
Maximilian Bachl\> 341455
\end{tabbing}
\section*{Aufgabe 1}
Es handelt sich hier um die Suche nach Extrema mit Nebenbedingung, weshalb wir zunächst nach Extrema auf dem Rand des Kreises suchen.\\\\
Die Nebenbedingung lautet: $g(x,y) = x^2 + y^2 - 1 = 0$.
\begin{enumerate}
\item {Singulärer Fall:}
\begin{align*}
\nabla g(\vec x) &= \vec 0 = \begin{pmatrix}2x \\ 2y\end{pmatrix}  \Rightarrow (x,y) = (0,0)
\intertext{Da $g(0,0) = -1 \neq 0$ gibt es hier keinen singulären Fall.}
\end{align*}
\item {$\nabla f( \vec x) = \lambda \cdot \nabla g(\vec x)$}
\begin{align*}
\nabla f(\vec x) &= \begin{pmatrix}6x-2y \\ -2x + 2y \end{pmatrix}= \lambda \begin{pmatrix}2x \\ 2y\end{pmatrix} = \lambda \cdot \nabla g(\vec x)
\intertext{Also:}
\text{I: }&6x-2y = \lambda \cdot 2x \Leftrightarrow 3- \frac{y}{x} = \lambda \Leftrightarrow -\frac{y}{x} = \lambda -3\\
\text{II: }&-2x + 2y = \lambda \cdot 2y \Leftrightarrow -\frac{x}{y} +1 = \lambda \Leftrightarrow -\frac{x}{y} = \lambda -1\\
\Leftrightarrow& \frac{y}{x} = -\frac{1}{\lambda -1}\\
\text{III: }&x^2 + y^2 -1 = 0\\
\end{align*}
Addiert man nun I und II so erhält man $0 = \lambda -3 -\frac{1}{\lambda -1} \Leftrightarrow 0 = \lambda^2 -4\lambda +2 \Leftrightarrow \lambda = 2 \pm \sqrt 2$.

Durch Einsetzen erhält man für $\lambda = 2 - \sqrt 2$, dass $x = -\frac{1}{2} \sqrt{2-\sqrt{2}}$ und $y = \frac{1}{2} \left(-3 \sqrt{2-\sqrt{2}}+\left(2-\sqrt{2}\right)^{\frac{3}{2}}\right)$.\\
Für $\lambda = 2 + \sqrt 2$ ist $x = \frac{1}{2} \sqrt{2-\sqrt{2}}$ und $y = \frac{1}{2} \left(3 \sqrt{2-\sqrt{2}}+\left(2-\sqrt{2}\right)^{\frac{3}{2}}\right)$. 

$$\vec x_{k_1} = \vectwo{-\frac{1}{2} \sqrt{2-\sqrt{2}}}{\frac{1}{2} \left(-3 \sqrt{2-\sqrt{2}}+\left(2-\sqrt{2}\right)^{\frac{3}{2}}\right)}$$
$$\vec x_{k_2} = \vectwo{\frac{1}{2} \sqrt{2-\sqrt{2}}}{\frac{1}{2} \left(3 \sqrt{2-\sqrt{2}}+\left(2-\sqrt{2}\right)^{\frac{3}{2}}\right)}$$

$$f(\vec x_{k_1}) = 2 + \sqrt 2$$
$$f(\vec x_{k_2}) = 14+9 \sqrt{2}$$
\end{enumerate}
Wir prüfen zunächst die notwendige Bedingung für kritische Punkte$\nabla f(\vec x_k) = 0$:
\begin{align*}
\nabla f( \vec x) &= \vec 0 = \begin{pmatrix}2\left(x-y\right) + 4x \\ -2\left(x-y\right) \end{pmatrix} = \begin{pmatrix} 6x -2y \\ -2x + 2y\end{pmatrix}
\intertext{Also:}
0 &= 6x-2y \\ % \text{Evtl. Nummerierung hinzufügen}\\
0 &= -2x + 2y \\
\Rightarrow 0 &= 4x \Rightarrow x=0 \Rightarrow y=0
\end{align*}
Wir erhalten deshalb einen kritischen Punkt: $x_{k3} = \begin{pmatrix}0 \\ 0\end{pmatrix}$.\\
Die Hessematrix ist, da es sich bei $f$ um eine zweimal stetig partiell differenzierbare Funktion handelt, gemäß dem Satz von Schwarz, symmetrisch.
\begin{align*}
f''(\vec x) &= H_f(\vec x) =  \begin{pmatrix}6 & -2 \\ -2 & 2\end{pmatrix}\\
D_1 &= \det (6) = 6 > 0\\
D_2 &= \det \begin{pmatrix} 6 & -2 \\ -2 & 2\end{pmatrix} = 8 > 0
\end{align*}
Damit ist $H_f(\vec x)$ positiv definit, woraus schlusszufolgern ist, dass die Funktion $f$ bei $f(0,0) = 0$ ein lokales Minimum besitzt. Da $f(x,y)$ eine Komposition aus $(x-y)^2 > 0 \; \; \forall x,y \in \mathbb{R}$ und $2x^2 > 0 \; \; \forall x \in \mathbb{R}$ ist, ist $f(0,0) = 0$ sogar ein globales Minimum.

Um nun über globale Extrema zu entscheiden, reicht es, die Funktionswerte von $f$ zu vergleichen, da die gegebene Menge kompakt ist und $f$ stetig ist.

Da $f(x_{k_1}) = 2 + \sqrt 2$, $f(x_{k_2}) = 14+9 \sqrt{2}$ und $f(x_{k_3}) = 0$, ist bei $x_{k_3}$ ein globales Minimum und bei $x_{k_2}$ ein globales Maximum unter der Beschränkung.

\textbf{\textsc{Müssen wir für $x_{k_1}$ noch irgendwas zeigen, oder reicht das? -- Max}}

\section*{Aufgabe 2}
Um lokale Extrema einer mehrdimensionalen Funktion zu bestimmen, müssen wir 1. die kritischen Punkte finden und 2. diese als Funktionswerte der Hessematrix übergeben.
\begin{enumerate}
	\item kritische Punkte sind alle Funktionswerte die $\nabla f(x,y,z) = \vec0$ erfüllen.
	\[\nabla f(x,y,z) = 
	\begin{pmatrix}
		4x \\ 
		4z + 2y \\
		4y + 10z	
	\end{pmatrix} \overset{\text{DGL}}{\Rightarrow}
	\begin{pmatrix}
		4 & 0 & 0 \\
		0 & 2 & 4 \\
		0 & 4 & 10 
	\end{pmatrix} \overset{\text{Gauss}}{\Rightarrow}
	\begin{pmatrix}
		1 & 0 & 0 \\
		0 & 1 & 0 \\
		0 & 0 & 1
	\end{pmatrix}\]
	Der Kern der Einheitsmatrix ist der Null-Vektor. Dies ist auch der einzige kritische Punkt demnach.
	\item Hesse-Matrix berechnen und kritische Punkte einfügen:
	\[H_f(x,y,z) =
	\begin{pmatrix}
		4 & 0 & 0 \\
		0 & 2 & 4 \\
		0 & 4 & 10
	\end{pmatrix}
	\] 
	Die Hessematrix ist, da es sich bei $f$ um eine zweimal stetig partiell differenzierbare Funktion handelt, gemäß dem Satz von Schwarz, symmetrisch.\\
	Da die Hesse Matrix konstant ist, müssen wir den Punkt offensichtlich nicht einsetzen. Mit dem Hurwitzkritirium kann nun überprüft, was der kritische Punkt nun ist. 
	\begin{align*}
	 	&\det(4) = 4 \\
	 	&\det \begin{pmatrix}
	 		4 & 0 \\
	 		0 & 2	
	 	\end{pmatrix} = 8 \\
	 	&\det \begin{pmatrix}
	 		4 & 0 & 0 \\
			0 & 2 & 4 \\
			0 & 4 & 10
	 	\end{pmatrix} \overset{\text{nach Laplace}}{=} 4\cdot (2\cdot10 - 4\cdot4) = 16
	 \end{align*} Damit ist $H_f(0,0,0)$ positiv definit. Somit ist ein lokales Minimum bei $\vec 0$. Ist es auch ein globales Minimum? 
	$2x^2$, $y^2$ und $5z^2$ werden nie negativ. Also kann nur $4yz$ negativ werden. Da aber $4yz  + y^2 + 5z^2 \ge 0$ für alle $(x,y) \in \mathbb{R}^2$ offensichtlich gilt, kann $f$ keine negativen Funktionswerte haben,  und muss somit, da $f(0,0,0) = 0$, an der Stelle $(0,0,0)$ ein globales Minimum haben.
\end{enumerate}
\section*{Aufgabe 3}
Wir haben die Nebenbedingung $g(a,b) = a + b -10 = 0$. Außerdem haben wir die Funktion $f(a,b) = ab$. Da der von g gebildete Körper kein Inneres besitzt, kann es in diesem auch keine Extrema geben.

Wir überprüfen, ob ein singulärer Fall vorliegt: 
\begin{align*}
\nabla g(a,b) &= \vectwo{1}{1} = \vectwo{0}{0} \text{ Da es hierfür keine Lösung gibt, liegt kein singulärer Fall vor.}
\end{align*}

Als nächstes überprüfen wir auf andere Extrema am Rand.
\begin{align*}
\nabla f(a,b) = \vectwo{b}{a} &= \lambda \vectwo{1}{1} = \lambda \nabla g(a,b) \\
\text{Es gilt: } a + b - 10 &= 0 \\
\text{Daraus folgt: }& a = \lambda \\
& b = \lambda \\
& \lambda = 5 \Rightarrow a = 5 \land b = 5
\end{align*}

Es gibt somit den zu untersuchenden kritischen Punkt $\vec x_{k_1} = \vectwo{5}{5}$
\begin{align*}
f''(a,b) = \begin{pmatrix}
0 & 1 \\
1 & 0 \\
\end{pmatrix}
\end{align*}
Da $det_1 = 0$ und $det_2 < 0$ ist die Hesse-Matrix semidefinit und somit kann man bei $\vectwo{5}{5}$ keine Aussage machen. Da man sich die Funktion $f(a,b) = ab$ aber gut vorstellen kann, weiß man, dass es keine Extrema gibt. 
Da $f$ bei $a$ und $b$ größer 0 s.m.w ist, muss es sich bei $\vectwo{5}{5}$ um ein globales Extremum (mit der Beschränkung $g$) handeln.


\end{document}