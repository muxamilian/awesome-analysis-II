\documentclass[10pt,a4paper,parskip=half]{scrartcl}
\usepackage[utf8]{inputenc}
\usepackage{amsmath}
\usepackage{amsfonts}
\usepackage{amssymb}
\usepackage{mathpazo}
\usepackage{stmaryrd} % Für den Widerspruchsblitz :D
\usepackage[a4paper,
left=3.0cm, right=3.0cm,
top=2.0cm, bottom=2.0cm]{geometry}
\usepackage{fullpage}
\usepackage[german]{babel}
\usepackage{enumerate}
\setlength{\unitlength}{1cm}
\newcommand{\N}{\mathbb{N}}
\newcommand{\A}{\mathcal{A}}
\newcommand{\R}{\mathbb{R}}
\parindent 0mm
\author{Tom}
\title{Analysis 2 - Hausaufgabe 7}

% new commands for vectors
\newcommand{\vectwo}[2]{\begin{pmatrix}#1\\#2\\\end {pmatrix}}
\newcommand{\vecthree}[3]{\begin{pmatrix}#1\\#2\\#3\\\end {pmatrix}}

\begin{document}
\begin{center}
\textsc{\Large{Analysis 2 - Hausaufgabe 6}} \\
\end{center}
\begin{tabbing}
Tom Nick \hspace{1.4cm}\= 342225\\
Tom Lehmann\> 340621\\
Maximilian Bachl\> 341455
\end{tabbing}
\section*{Aufgabe 1}
Da der Definitionsbereich von $f$ offen und konvex ist, muss nur noch überprüft werden, ob $\text{rot}\ \vec v = \vec 0$

\begin{align*}
\text{rot}\ \vec v &= \text{rot}\ \vecthree{f(x,y,z)}{x^2 + yz^2}{y^2z}\\
&= \vecthree{2yz - 2zy}{-(0-\frac{\partial f}{\partial z})}{2x - \frac{\partial f}{\partial y}}\\
&= \vecthree{0}{\frac{\partial f}{\partial z}}{2x - \frac{\partial f}{\partial y}}
\end{align*}

Damit $\text{rot}(\vec v) = \vec 0$ muss gelten, dass $f$ kein $z$ enthält (da sonst $\frac{\partial f}{\partial z}$ nicht 0 ist), und dass es nur ein $y$ enthält. Somit muss $f(x,y,z) = 2xy + K(x)$, wobei $K(x)$ eine von $y$ und $z$ unabhängige Konstante ist.

Daher muss gelten:
\begin{align*}
\text{grad}\ u &= - \vec v \\
\Leftrightarrow \vecthree{\frac{\partial u}{\partial x}}{\frac{\partial u}{\partial y}}{\frac{\partial u}{\partial z}} &= - \vecthree{2xy + K(x)}{x^2 + yz^2}{y^2z}\begin{array}{l} \text{I} \\ \text{II} \\ \text{III}\end{array}
\end{align*}
Daraus lassen sich zeilenweise Gleichungen aufstellen.
\begin{align*}
\text{I: }\frac{\partial u}{\partial x} &= - 2xy + K(x)\\
\text{II: }\frac{\partial u}{\partial y} &= -x^2 - yz^2\\
\text{III: }\frac{\partial u}{\partial z} &= -y^2z\\
\text{Wir integrieren I nach x: }&\\
u &= -x^2y - \int K(x) \partial x + C_1(y,z)\\
\text{Wir integrieren II nach $y$: }&\\
u &= -x^2y - \frac{y^2z^2}{2} + C_2(x,z)\\
\text{Wir integrieren III nach $z$: }&\\
u &= -\frac{y^2z^2}{2} + C_3(x,y)\\
\text{Man sieht, dass $u$ folgende Gestalt hat: }\\
u &= -x^2y - \int K(x) \partial x - \frac{y^2z^2}{2} + C
\end{align*}
Bleibt das Ergebnis noch zu überprüfen:
\begin{align*}
\text{grad}\ \vecthree{\frac{\partial u}{\partial x}}{\frac{\partial u}{\partial y}}{\frac{\partial u}{\partial z}} &= \vecthree{-2xy - K(x)}{-x^2-yz^2}{-y^2z}\\
&= -\vecthree{2xy + K(x)}{x^2+yz^2}{y^2z} = -\vec v
\end{align*}
Somit ist u das Potential von $\vec v$, falls dieses existiert (nach geeigneter Wahl von $f$).
\section*{Aufgabe 2}
Sei $\vec v(x,y,z) = \vecthree{v_1}{v_2}{v_3}$ und $\vec w(x,y,z) = \vecthree{w_1}{w_2}{w_3}$. 
\begin{align*}
&\text{div}\left(\vecthree{v_1}{v_2}{v_3} \times \vecthree{w_1}{w_2}{w_3}\right) = \left( \text{rot} \vecthree{v_1}{v_2}{v_3} \right) \cdot \vecthree{w_1}{w_2}{w_3} - \vecthree{v_1}{v_2}{v_3} \cdot \text{rot}\left( \vecthree{w_1}{w_2}{w_3} \right) \\
\Leftrightarrow\; &\text{div}\left(\vecthree{v_2w_3 - v_3w_2}{-v_1w_3 + v_3w_1}{v_1w_2 - v_2w_1}\right) = \vecthree{\frac{\partial v_3}{\partial y} - \frac{\partial v_2}{\partial z}}{- \frac{\partial v_3}{\partial x} + \frac{\partial v_1}{\partial z}}{\frac{\partial v_2}{\partial x} - \frac{\partial v_1}{\partial y}} \cdot \vecthree{w_1}{w_2}{w_3} - \vecthree{v_1}{v_2}{v_3} \cdot \vecthree{\frac{\partial w_3}{\partial y} - \frac{\partial w_2}{\partial z}}{- \frac{\partial w_3}{\partial x} + \frac{\partial w_1}{\partial z}}{\frac{\partial w_2}{\partial x} - \frac{\partial w_1}{\partial y}} \\
\Leftrightarrow\; &\frac{\partial }{\partial x}(v_2w_3 - v_3w_2) + \frac{\partial }{\partial y}(-v_1w_3 + v_3w_1) + \frac{\partial }{\partial z}(v_1w_2 - v_2w_1)\\
&= \left(\frac{\partial }{\partial y}v_3\right)w_1 - \left(\frac{\partial }{\partial z}v_2\right)w_1 - \left(\frac{\partial }{\partial x}v_3\right)w_2 + \left(\frac{\partial }{\partial z}v_1\right)w_2 + \left(\frac{\partial }{\partial x}v_2\right)w_3 - \left(\frac{\partial }{\partial y}v_1\right)w_3\\
 &- v_1\left(\frac{\partial }{\partial y}w_3\right) + v_1\left(\frac{\partial }{\partial z}w_2\right) + v_2\left(\frac{\partial }{\partial x}w_3\right) - v_2\left(\frac{\partial }{\partial z}w_1\right) - v_3\left(\frac{\partial }{\partial x}w_2\right) + v_3\left(\frac{\partial }{\partial y}w_1\right) \\
\Leftrightarrow\; &\frac{\partial }{\partial x}v_2w_3 - \frac{\partial }{\partial x}v_3w_2 - \frac{\partial }{\partial y}v_1w_3 + \frac{\partial }{\partial y}v_3w_1 + \frac{\partial }{\partial z}v_1w_2 - \frac{\partial }{\partial z}v_2w_1\\
&= \left(\frac{\partial }{\partial y}v_3\right)w_1 - \left(\frac{\partial }{\partial z}v_2\right)w_1 - \left(\frac{\partial }{\partial x}v_3\right)w_2 + \left(\frac{\partial }{\partial z}v_1\right)w_2 + \left(\frac{\partial }{\partial x}v_2\right)w_3 - \left(\frac{\partial }{\partial y}v_1\right)w_3\\
 &- v_1\left(\frac{\partial }{\partial y}w_3\right) + v_1\left(\frac{\partial }{\partial z}w_2\right) + v_2\left(\frac{\partial }{\partial x}w_3\right ) - v_2\left(\frac{\partial }{\partial z}w_1\right) - v_3\left(\frac{\partial }{\partial x}w_2\right) + v_3\left(\frac{\partial }{\partial y}w_1\right) \\
\stackrel{\text{prod. R.}}{\Leftrightarrow}\; &\left(\frac{\partial }{\partial x}v_2\right)w_3 +  \left(\frac{\partial }{\partial x}w_3\right)w_2 - \left(\left(\frac{\partial }{\partial x}v_3\right)w_2 + \left(\frac{\partial }{\partial x}w_2\right)v_3\right)  - \left(\left(\frac{\partial }{\partial y}v_1\right)w_3 + \left(\frac{\partial }{\partial y}w_3\right)v_1\right) \\
& + \left(\frac{\partial }{\partial y}v_3\right)w_1 + \left(\frac{\partial }{\partial y}w_1\right)v_3 + \left(\frac{\partial }{\partial z}v_1\right)w_2 + \left(\frac{\partial }{\partial z}w_2\right)v_1 - \left( \left(\frac{\partial }{\partial z}v_2\right)w_1 + \left(\frac{\partial }{\partial z}w_1\right)v_2 \right)\\
&= \left(\frac{\partial }{\partial y}v_3\right)w_1 - \left(\frac{\partial }{\partial z}v_2\right)w_1 - \left(\frac{\partial }{\partial x}v_3\right)w_2 + \left(\frac{\partial }{\partial z}v_1\right)w_2 + \left(\frac{\partial }{\partial x}v_2\right)w_3 - \left(\frac{\partial }{\partial y}v_1\right)w_3\\
 &- v_1\left(\frac{\partial }{\partial y}w_3\right) + v_1\left(\frac{\partial }{\partial z}w_2\right) + v_2\left(\frac{\partial }{\partial x}w_3\right ) - v_2\left(\frac{\partial }{\partial z}w_1\right) - v_3\left(\frac{\partial }{\partial x}w_2\right) + v_3\left(\frac{\partial }{\partial y}w_1\right) \\
\Leftrightarrow\; &\left(\frac{\partial }{\partial x}v_2\right)w_3 +  \left(\frac{\partial }{\partial x}w_3\right)w_2 - \left(\frac{\partial }{\partial x}v_3\right)w_2 - \left(\frac{\partial }{\partial x}w_2\right)v_3 - \left(\frac{\partial }{\partial y}v_1\right)w_3 - \left(\frac{\partial }{\partial y}w_3\right)v_1 \\
& + \left(\frac{\partial }{\partial y}v_3\right)w_1 + \left(\frac{\partial }{\partial y}w_1\right)v_3 + \left(\frac{\partial }{\partial z}v_1\right)w_2 + \left(\frac{\partial }{\partial z}w_2\right)v_1 - \left(\frac{\partial }{\partial z}v_2\right)w_1 - \left(\frac{\partial }{\partial z}w_1\right)v_2\\
&= \left(\frac{\partial }{\partial y}v_3\right)w_1 - \left(\frac{\partial }{\partial z}v_2\right)w_1 - \left(\frac{\partial }{\partial x}v_3\right)w_2 + \left(\frac{\partial }{\partial z}v_1\right)w_2 + \left(\frac{\partial }{\partial x}v_2\right)w_3 - \left(\frac{\partial }{\partial y}v_1\right)w_3\\
 &- v_1\left(\frac{\partial }{\partial y}w_3\right) + v_1\left(\frac{\partial }{\partial z}w_2\right) + v_2\left(\frac{\partial }{\partial x}w_3\right ) - v_2\left(\frac{\partial }{\partial z}w_1\right) - v_3\left(\frac{\partial }{\partial x}w_2\right) + v_3\left(\frac{\partial }{\partial y}w_1\right) \\
\end{align*}
\section*{Aufgabe 3}
\begin{enumerate}[(a)]
\item
\begin{align*}
\text{div}\ \text{rot}\ \vec v &= \text{div}\ \text{rot}\ \vecthree{x^2 - y^2}{y^2 -z^2}{z^2 - x^2}\\
&= \text{div}\ \vecthree{2z}{2x}{2y}\\
&= 0 + 0 + 0 \\
&= 0\\
\\
\text{rot}\ \text{grad}\ f &= \text{rot}\ \text{grad}\ e^{xy} \\
&= \text{rot}\ \vecthree{ye^{xy}}{xe^{xy}}{0} \\
&= \vecthree{0 -0}{0-0}{e^{xy} + xye^{xy} - e^{xy} - xye^{xy}} \\
&= \vecthree{0}{0}{0}
\end{align*}
\item
\begin{align*}
\text{div}\ \text{grad}\ f &= \text{div}\ \text{grad}\ e^{xy}\\
&= \text{div}\ \vecthree{ye^{xy}}{xe^{xy}}{0}\\
&= y^2e^{xy} + x^2e^{xy}\\
\\
\Delta f &= \text{div}\ \text{grad}\ f\\
&\overset{\text{siehe oben}}{=} y^2e^{xy} + x^2e^{xy}
\end{align*}
\item
\begin{align*}
fg &= xe^{xy} + y^2e^{xy} + z^3e^{xy} \\
f \vec v &= \vecthree{x^2e^{xy}-y^2e^{xy}}{y^2e^{xy}-z^2e^{xy}}{z^2e^{xy}-x^2e^{xy}} \\
\\
\text{grad}\ fg &= \text{grad}\ xe^{xy} + y^2e^{xy} + z^3e^{xy} \\
&= \vecthree{e^{xy} + xye^{xy} + y^3e^{xy} + yz^3e^{xy}}{x^2e^{xy} + 2ye^{xy} + y^2xe^{xy} + xz^3e^{xy}}{3z^2e^{xy}}\\
\\
\text{div}\ f\vec v &= \text{div}\ \vecthree{x^2e^{xy}-y^2e^{xy}}{y^2e^{xy}-z^2e^{xy}}{z^2e^{xy}-x^2e^{xy}}\\
&= 2xe^{xy} + yx^2e^{xy} - y^3e^{xy} + 2ye^{xy} + y^2xe^{xy} - z^2xe^{xy} + 2ze^{xy}\\
\\
\text{rot}\ f\vec v &= \text{rot}\ \vecthree{x^2e^{xy}-y^2e^{xy}}{y^2e^{xy}-z^2e^{xy}}{z^2e^{xy}-x^2e^{xy}}\\
&= \vecthree{z^2xe^{xy} - x^3e^{xy} + 2ze^{xy}}{-z^2ye^{xy} + 2xe^{xy} + x^2ye^{xy}}{y^3e^{xy} - z^2ye^{xy} - (x^3e^{xy} - 2ye^{xy} - y^2xe^{xy})}\\
&\text{Bitte nochmal nachrechnen.}
\end{align*}
\end{enumerate}
\end{document}