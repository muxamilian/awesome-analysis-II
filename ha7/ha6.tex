\documentclass[10pt,a4paper,parskip=half]{scrartcl}
\usepackage[utf8]{inputenc}
\usepackage{amsmath}
\usepackage{amsfonts}
\usepackage{amssymb}
\usepackage{mathpazo}
\usepackage{stmaryrd} % Für den Widerspruchsblitz :D
\usepackage[a4paper,
left=3.0cm, right=3.0cm,
top=2.0cm, bottom=2.0cm]{geometry}
\usepackage{fullpage}
\usepackage[german]{babel}
\usepackage{enumerate}
\setlength{\unitlength}{1cm}
\newcommand{\N}{\mathbb{N}}
\newcommand{\A}{\mathcal{A}}
\newcommand{\R}{\mathbb{R}}
\parindent 0mm
\author{Tom}
\title{Analysis 2 - Hausaufgabe 7}

% new commands for vectors
\newcommand{\vectwo}[2]{\begin{pmatrix}#1\\#2\\\end {pmatrix}}
\newcommand{\vecthree}[3]{\begin{pmatrix}#1\\#2\\#3\\\end {pmatrix}}

\begin{document}
\begin{center}
\textsc{\Large{Analysis 2 - Hausaufgabe 6}} \\
\end{center}
\begin{tabbing}
Tom Nick \hspace{1.4cm}\= 342225\\
Tom Lehmann\> 340621\\
Maximilian Bachl\> 341455
\end{tabbing}
\section*{Aufgabe 1}
\section*{Aufgabe 2}
Sei $\vec v(x,y,z) = \vecthree{v_1}{v_2}{v_3}$ und $\vec w(x,y,z) = \vecthree{w_1}{w_2}{w_3}$. 
\begin{align*}
&\text{div}\left(\vecthree{v_1}{v_2}{v_3} \times \vecthree{w_1}{w_2}{w_3}\right) = \left( \text{rot} \vecthree{v_1}{v_2}{v_3} \right) \cdot \vecthree{w_1}{w_2}{w_3} - \vecthree{v_1}{v_2}{v_3} \cdot \text{rot}\left( \vecthree{w_1}{w_2}{w_3} \right) \\
\Leftrightarrow\; &\text{div}\left(\vecthree{v_2w_3 - v_3w_2}{-v_1w_3 + v_3w_1}{v_1w_2 - v_2w_1}\right) = \vecthree{\frac{\partial v_3}{\partial y} - \frac{\partial v_2}{\partial z}}{- \frac{\partial v_3}{\partial x} + \frac{\partial v_1}{\partial z}}{\frac{\partial v_2}{\partial x} - \frac{\partial v_1}{\partial y}} \cdot \vecthree{w_1}{w_2}{w_3} - \vecthree{v_1}{v_2}{v_3} \cdot \vecthree{\frac{\partial w_3}{\partial y} - \frac{\partial w_2}{\partial z}}{- \frac{\partial w_3}{\partial x} + \frac{\partial w_1}{\partial z}}{\frac{\partial w_2}{\partial x} - \frac{\partial w_1}{\partial y}} \\
&\text{Ich glaube ab da ist es schon falsch.}\\
\Leftrightarrow\; &\frac{\partial }{\partial x}(v_2w_3 - v_3w_2) + \frac{\partial }{\partial y}(-v_1w_3 + v_3w_1) + \frac{\partial }{\partial z}(v_1w_2 - v_2w_1)\\
&= \frac{\partial }{\partial y}v_3w_1 - \frac{\partial }{\partial z}v_2w_1 - \frac{\partial }{\partial x}v_3w_2 + \frac{\partial }{\partial z}v_1w_2 + \frac{\partial }{\partial x}v_2w_3 - \frac{\partial }{\partial y}v_1w_3\\
 &- v_1\frac{\partial }{\partial y}w_3 + v_1\frac{\partial }{\partial z}w_2 + v_2\frac{\partial }{\partial x}w_3 - v_2\frac{\partial }{\partial z}w_1 - v_3\frac{\partial }{\partial x}w_2 + v_3\frac{\partial }{\partial y}w_1\\
\Leftrightarrow\; &\frac{\partial }{\partial x}v_2w_3 - \frac{\partial }{\partial x}v_3w_2 - \frac{\partial }{\partial y}v_1w_3 + \frac{\partial }{\partial y}v_3w_1 + \frac{\partial }{\partial z}v_1w_2 - \frac{\partial }{\partial z}v_2w_1\\
&= \frac{\partial }{\partial y}v_3w_1 - \frac{\partial }{\partial z}v_2w_1 - \frac{\partial }{\partial x}v_3w_2 + \frac{\partial }{\partial z}v_1w_2 + \frac{\partial }{\partial x}v_2w_3 - \frac{\partial }{\partial y}v_1w_3\\
 &- v_1\frac{\partial }{\partial y}w_3 + v_1\frac{\partial }{\partial z}w_2 + v_2\frac{\partial }{\partial x}w_3 - v_2\frac{\partial }{\partial z}w_1 - v_3\frac{\partial }{\partial x}w_2 + v_3\frac{\partial }{\partial y}w_1\\
\Leftrightarrow\; & 0\\
&= - v_1\frac{\partial }{\partial y}w_3 + v_1\frac{\partial }{\partial z}w_2 + v_2\frac{\partial }{\partial x}w_3 - v_2\frac{\partial }{\partial z}w_1 - v_3\frac{\partial }{\partial x}w_2 + v_3\frac{\partial }{\partial y}w_1\\
&\text{Irgendwie geht das nicht.}
\end{align*}
\section*{Aufgabe 3}

\end{document}