\documentclass[10pt,a4paper,parskip=half]{scrartcl}
\usepackage[utf8]{inputenc}
\usepackage{amsmath}
\usepackage{amsfonts}
\usepackage{amssymb}
\usepackage{mathpazo}
\usepackage[a4paper,
left=3.0cm, right=3.0cm,
top=2.0cm, bottom=2.0cm]{geometry}
\usepackage{fullpage}
\usepackage[german]{babel}
\usepackage{enumerate}
\setlength{\unitlength}{1cm}
\newcommand{\N}{\mathbb{N}}
\newcommand{\A}{\mathcal{A}}
\newcommand{\R}{\mathbb{R}}
\parindent 0mm
\author{Tom}
\title{Analysis 2 - Hausaufgabe 4}

% new commands for vectors
\newcommand{\vectwo}[2]{\left(\begin {array} {c}#1\\#2\\\end {array} \right)}
\newcommand{\vecthree}[3]{\left(\begin {array} {c}#1\\#2\\#3\\\end {array} \right) }

\begin{document}
\begin{center}
\textsc{\Large{Analysis 2 - Hausaufgabe 4}} \\
\end{center}
\begin{tabbing}
Tom Nick \hspace{1.4cm}\= 342225\\
Tom Lehmann\> 340621\\
Maximilian Bachl\> 341455
\end{tabbing}
\section{Aufgabe}
\subsubsection*{\textbf{(a)}}
\subsubsection*{\textbf{(b)}}
\begin{align*}
\frac{\partial \vec f}{\partial r}(r, \varphi) &= \begin{pmatrix}\cos(\varphi) \\ \sin(\varphi)\end{pmatrix}\\
\frac{\partial \vec f}{\partial \varphi}(r,\varphi) &= \begin{pmatrix}-r\sin(\varphi) \\ r \cos(\varphi)\end{pmatrix} 
\end{align*}
\subsubsection*{\textbf{(c)}}
Die Ableitungsmatrix $\vec {f'}(r ,\varphi)$ ist demnach:
\begin{align*}
\vec {f'}(r, \varphi) &= \begin{pmatrix}\cos(\varphi) & -r \sin(\varphi) \\ \sin(\varphi) & r \cos(\varphi)\end{pmatrix}
\intertext{Für die Determinante gilt deshalb:}
\det(\vec {f'}(r, \varphi)) &= \det \begin{pmatrix}\cos(\varphi) & -r \sin(\varphi) \\ \sin(\varphi) & r \cos(\varphi)\end{pmatrix}\\
&= r\cos^2(\varphi) + r \sin^2(\varphi)\\
&= r \left( \cos^2(\varphi) + \sin^2(\varphi)\right)\\
&= r
\end{align*}
\section{Aufgabe}
\subsubsection*{\textbf{(a)}}
Die Komposition $g \circ g$ ist nicht erklärt, da der Urbildraum Elemente des $\mathbb{\R}^3$ verlangt, die Funktionswerte von $g$ jedoch nur aus $\mathbb{\R}$ sind.\\
Die Komposition $g \circ \vec f$ ist erklärt, da sowohl die Funktionswerte von $f$ als auch der Urbildraum von $g$ Elemente des $\mathbb{\R}^3$ sind.\\
Die Komposition $\vec f \circ g$ ist erklärt, da sowohl die Funktionswerte von $g$ als auch der Urbildraum von $f$ Elemente des $\mathbb{\R}$ sind.\\
Die Komposition $\vec f \circ \vec f$ ist nicht erklärt, da der Urbildraum Elemente des $\mathbb{\R}$ verlangt, die Funktionswerte von $\vec f$ jedoch aus $\mathbb{\R}^3$ sind.

\end{document}