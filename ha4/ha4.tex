\documentclass[10pt,a4paper,parskip=half]{scrartcl}
\usepackage[utf8]{inputenc}
\usepackage{amsmath}
\usepackage{amsfonts}
\usepackage{amssymb}
\usepackage{mathpazo}
\usepackage[a4paper,
left=3.0cm, right=3.0cm,
top=2.0cm, bottom=2.0cm]{geometry}
\usepackage{fullpage}
\usepackage[german]{babel}
\usepackage{enumerate}
\setlength{\unitlength}{1cm}
\newcommand{\N}{\mathbb{N}}
\newcommand{\A}{\mathcal{A}}
\newcommand{\R}{\mathbb{R}}
\parindent 0mm
\author{Tom}
\title{Analysis 2 - Hausaufgabe 3}

% new commands for vectors
\newcommand{\vectwo}[2]{\left(\begin {array} {c}#1\\#2\\\end {array} \right)}
\newcommand{\vecthree}[3]{\left(\begin {array} {c}#1\\#2\\#3\\\end {array} \right) }

% Füllt man nach Spalte und dann nach Zeile. Dann kann man besser von Vektoren kopieren
\newcommand{\mattwotwo}[4]{\left(\begin {array} {cc}#1 & #3\\#2 & #4\\\end {array} \right)}
\newcommand{\dettwotwo}[4]{\left|\begin {array} {cc}#1 & #3\\#2 & #4\\\end {array} \right|}

\begin{document}
\begin{center}
\textsc{\Large{Analysis 2 - Hausaufgabe 3}} \\
\end{center}
\begin{tabbing}
Tom Nick \hspace{1.4cm}\= 342225\\
Tom Lehmann\> 340621\\
Maximilian Bachl\> 341455
\end{tabbing}
\section*{Aufgabe 1}
\begin{enumerate}[(a)]
\item
Wir haben dann eine Scheibe. Ich bin aber gerade zu faul eine zu malen...
\item
$$ \frac{\partial \vec f}{\partial r} = \vectwo{\cos(\varphi)}{\sin(\varphi)}$$
$$ \frac{\partial \vec f}{\partial \varphi} = \vectwo{-r\sin(\varphi)}{\cos(\varphi)}$$

Zum Zeichnen hab ich wieder keine Lust. bei (a) ein Kreis und bei (b) kann ich es mir gerade nicht vorstellen.
\item
$$ \vec f'(\vectwo{r}{\varphi}) = \mattwotwo{\cos(\varphi)}{\sin(\varphi)}{-r\sin(\varphi)}{\cos(\varphi)} $$
$$\dettwotwo{\cos(\varphi)}{\sin(\varphi)}{-r\sin(\varphi)}{\cos(\varphi)} = \cos^2(\varphi) + r\sin^2(\varphi)$$
\end{enumerate}
\section*{Aufgabe 2}
\begin{enumerate}[(a)]
\item
$\vec f \circ g$ ist die einzige mögliche Komposition, da nur so der Bildraum der inneren Funktion auf den Urbildraum der äußeren abbildet.
\item
$$ (\vec f \circ g)' = (\vec f' \circ g) \cdot g'$$
$$ \vec f' = ... \text{keine Lust mehr} $$
\item
\end{enumerate}
\section*{Aufgabe 3}
\end{document}