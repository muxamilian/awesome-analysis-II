\documentclass[10pt,a4paper,parskip=half]{scrartcl}
\usepackage[utf8]{inputenc}
\usepackage{amsmath}
\usepackage{amsfonts}
\usepackage{amssymb}
\usepackage{mathpazo}
\usepackage{tikz}
\usetikzlibrary{patterns}
\usepackage{stmaryrd} % Für den Widerspruchsblitz :D
\usepackage[a4paper,
left=3.0cm, right=3.0cm,
top=2.0cm, bottom=2.0cm]{geometry}
\usepackage{fullpage}
\usepackage[german]{babel}
\usepackage{enumerate}
\setlength{\unitlength}{1cm}
\newcommand{\N}{\mathbb{N}}
\newcommand{\A}{\mathcal{A}}
\newcommand{\R}{\mathbb{R}}
\parindent 0mm
\author{Tom}
\title{Analysis 2 - Hausaufgabe 9}

% new commands for vectors
\newcommand{\vectwo}[2]{\begin{pmatrix}#1\\#2\\\end {pmatrix}}
\newcommand{\vecthree}[3]{\begin{pmatrix}#1\\#2\\#3\\\end {pmatrix}}

\usepackage{listings}
\usepackage{courier}
 \lstset{
         basicstyle=\footnotesize\ttfamily, % Standardschrift
         %numbers=left,               % Ort der Zeilennummern
         numberstyle=\tiny,          % Stil der Zeilennummern
         %stepnumber=2,               % Abstand zwischen den Zeilennummern
         numbersep=5pt,              % Abstand der Nummern zum Text
         tabsize=2,                  % Groesse von Tabs
         extendedchars=true,         %
         breaklines=true,            % Zeilen werden Umgebrochen
         keywordstyle=\color{red},
    		frame=b,         
 %        keywordstyle=[1]\textbf,    % Stil der Keywords
 %        keywordstyle=[2]\textbf,    %
 %        keywordstyle=[3]\textbf,    %
 %        keywordstyle=[4]\textbf,   \sqrt{\sqrt{}} %
         stringstyle=\color{white}\ttfamily, % Farbe der String
         showspaces=false,           % Leerzeichen anzeigen ?
         showtabs=false,             % Tabs anzeigen ?
         xleftmargin=17pt,
         framexleftmargin=17pt,
         framexrightmargin=5pt,
         framexbottommargin=4pt,
         %backgroundcolor=\color{lightgray},
         showstringspaces=false      % Leerzeichen in Strings anzeigen ?        
 }
 \usepackage{caption}
\DeclareCaptionFont{white}{\color{white}}
\DeclareCaptionFormat{listing}{\colorbox[cmyk]{0.43, 0.35, 0.35,0.01}{\parbox{\textwidth}{\hspace{15pt}#1#2#3}}}
\captionsetup[lstlisting]{format=listing,labelfont=white,textfont=white, singlelinecheck=false, margin=0pt, font={bf,footnotesize}}

\usepackage{color}
\usepackage{enumerate}



\begin{document}
\begin{center}
\textsc{\Large{Analysis 2 - Hausaufgabe 9}} \\
\end{center}
\begin{tabbing}
Tom Nick \hspace{1.4cm}\= 342225\\
Tom Lehmann\> 340621\\
Maximilian Bachl\> 341455
\end{tabbing}
\section*{Aufgabe 1}
\begin{enumerate}[(a)]
\item \begin{align*}
x &= r \sin \left( \vartheta \right) \cos (\varphi) \\
y &= r \sin (\vartheta ) \sin( \varphi) \\
z &= r \cos( \vartheta )
\intertext{\textbf{Ein Längenkreis ist ein verfickter Vollkreis! Ein Meridian  ist ein Halbkreis. Egal. Kackspassten! - Tom}}
 \vec{\gamma_1}(t) &= \begin{pmatrix}6300 \sin (t) \cos(0) \\ 6300  \sin(t) \sin(0) \\ 6300 \cos(t) \end{pmatrix} = \begin{pmatrix} 6300 \sin(t) \\ 0 \\ 6300 \cos(t) \end{pmatrix}  \quad  t \in [0,\pi]
\end{align*}
\item
\begin{align*}
\vec{\gamma_2}(t) &= t^2 \quad t \in [-1,1]
\end{align*}
\item
\begin{align*}
\vec{\gamma_3}(t) &= \begin{pmatrix} \sin(t) \cos(\tfrac{\pi}4 \\ \sin(t) \sin(\tfrac{\pi}4 \\ \cos(t)\end{pmatrix} \quad  t \in [0,2\pi[
\end{align*}
\end{enumerate}
\section*{Aufgabe 2}
\begin{align*}
\int\limits_{\gamma} \vec v \; d\vec s &= \int\limits_{-\pi}^{\pi} \vec v \left(\vec{\gamma}\left(t\right)\right) \cdot \dot{\gamma}(t) \; dt\\
&= \int\limits_{-\pi}^{\pi} \vec v\left(\cos(t),\cos(t),\sin(t) \right) \cdot \begin{pmatrix}-\sin(t) \\ -\sin(t) \\ \cos(t)\end{pmatrix}dt \\
&= \int\limits_{-\pi}^{\pi} \begin{pmatrix}\cos^2(t) + \sin^2(t) + \cos(t)\sin(t) \\ \frac{\cos(t)}{\cos^2(t)+ \sin^2(t)} + \cos(t)\sin(t) \\ \frac{\sin(t)}{\cos^2(t) + \sin^2(t)} + \cos(t) \cos(t) + 2 \cos(t)\sin(t)\end{pmatrix} \cdot \begin{pmatrix}-\sin(t) \\ -\sin(t) \\ \cos(t)\end{pmatrix}dt \\
\begin{split} &= \int\limits_{-\pi}^{\pi} -\sin(t) \cdot \left( \cos^2(t) + \sin^2(t) + \cos(t)\sin(t)  \right) -\sin(t) \cdot \left( \tfrac{\cos(t)}{\cos^2(t)+ \sin^2(t)} + \cos(t)\sin(t) \right)
\\&\quad + \cos(t) \cdot \left( \tfrac{\sin(t)}{\cos^2(t) + \sin^2(t)} + \cos^2(t) + 2 \cos(t)\sin(t)\right) dt \end{split}
\intertext{Da im Integral in jedem Summanden cos- oder sin-Terme ungerader Potenz vorhanden sind, ist das Integral  über eine ganze  Periodenlänge  von  $-\pi$ bis $\pi$ gerade $0$.} 
\end{align*}
TODO $\int_{\gamma} \vec{\omega} d\vec s$
\end{document}


