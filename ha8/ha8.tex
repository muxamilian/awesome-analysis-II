\documentclass[10pt,a4paper,parskip=half]{scrartcl}
\usepackage[utf8]{inputenc}
\usepackage{amsmath}
\usepackage{amsfonts}
\usepackage{amssymb}
\usepackage{mathpazo}
\usepackage{stmaryrd} % Für den Widerspruchsblitz :D
\usepackage[a4paper,
left=3.0cm, right=3.0cm,
top=2.0cm, bottom=2.0cm]{geometry}
\usepackage{fullpage}
\usepackage[german]{babel}
\usepackage{enumerate}
\setlength{\unitlength}{1cm}
\newcommand{\N}{\mathbb{N}}
\newcommand{\A}{\mathcal{A}}
\newcommand{\R}{\mathbb{R}}
\parindent 0mm
\author{Tom}
\title{Analysis 2 - Hausaufgabe 7}

% new commands for vectors
\newcommand{\vectwo}[2]{\begin{pmatrix}#1\\#2\\\end {pmatrix}}
\newcommand{\vecthree}[3]{\begin{pmatrix}#1\\#2\\#3\\\end {pmatrix}}

\begin{document}
\begin{center}
\textsc{\Large{Analysis 2 - Hausaufgabe 8}} \\
\end{center}
\begin{tabbing}
Tom Nick \hspace{1.4cm}\= 342225\\
Tom Lehmann\> 340621\\
Maximilian Bachl\> 341455
\end{tabbing}

\subsection*{Aufgabe 1}
	\begin{enumerate}[(a)]
		\item Wir wissen bereits, dass $\vec{v}$ wirbelfrei ist und ein Potential besitzt (Da $\vec{u}$ ein Potential ist. Damit $\vec{u} + \text{grad} f$ ein Vektorpotential von $\vec{v}$ ist, muss gelten:
		\begin{enumerate}[1.]
			\item $\nabla (\vec{u} + \text{grad} f) = -\vec{v}  $ \\ 
			\begin{align*}
				\nabla (\vec{u} + \text{grad} f) &= \nabla \vec{u} + \nabla (\text{grad} f) \\
				&= -\vec{v} + \nabla(\text{grad} f) \\
				&\Leftrightarrow \nabla(\text{grad} f) = 0 \\
			\end{align*}
			
		\end{enumerate}
	\end{enumerate}

\end{document}