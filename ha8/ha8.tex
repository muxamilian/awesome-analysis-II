\documentclass[10pt,a4paper,parskip=half]{scrartcl}
\usepackage[utf8]{inputenc}
\usepackage{amsmath}
\usepackage{amsfonts}
\usepackage{amssymb}
\usepackage{mathpazo}
\usepackage{stmaryrd} % Für den Widerspruchsblitz :D
\usepackage[a4paper,
left=3.0cm, right=3.0cm,
top=2.0cm, bottom=2.0cm]{geometry}
\usepackage{fullpage}
\usepackage[german]{babel}
\usepackage{enumerate}
\setlength{\unitlength}{1cm}
\newcommand{\N}{\mathbb{N}}
\newcommand{\A}{\mathcal{A}}
\newcommand{\R}{\mathbb{R}}
\parindent 0mm
\author{Tom}
\title{Analysis 2 - Hausaufgabe 8}

% new commands for vectors
\newcommand{\vectwo}[2]{\begin{pmatrix}#1\\#2\\\end {pmatrix}}
\newcommand{\vecthree}[3]{\begin{pmatrix}#1\\#2\\#3\\\end {pmatrix}}

\begin{document}
\begin{center}
\textsc{\Large{Analysis 2 - Hausaufgabe 8}} \\
\end{center}
\begin{tabbing}
Tom Nick \hspace{1.4cm}\= 342225\\
Tom Lehmann\> 340621\\
Maximilian Bachl\> 341455
\end{tabbing}
\section*{Aufgabe 1}
\subsubsection*{\textbf{(b)}}
\subsubsection*{(i)}
Da $\vec v$ stetig differenzierbar und $D$ konvex sowie offen ist, muss nur noch geprüft werden ob gilt:
\begin{align*}
\operatorname{div} ( \vec v ) &= 0\\
&= \cos(x) + \frac 23(x+y)^{-\frac 13} - \cos (x) - \frac 23(x+y)^{-\frac 13} = 0\\
\end{align*}
Damit ist gezeigt, dass ein Vektorpotential für $\vec v$ existiert.
\subsubsection*{(ii)}
Da $\vec v$ stetig differenzierbar ist und $D$ konvex sowie offen ist, muss nur noch geprüft werden ob gilt:
\begin{align*}
\text{rot} \vec v &= \vec 0\\
&= \begin{pmatrix}
\left(-\cos(x)z-\frac 23(x+y)^{-\frac 13}z \right)\frac {\partial}{\partial y} - (x+y)^{\frac 23} \frac{\partial}{\partial z} \\
\left(z^2  + \sin(x)\right) \frac{\partial}{\partial z} -\left(-\cos(x)z-\frac 23(x+y)^{-\frac 13}z \right) \frac{\partial}{\partial x}\\
(x+y)^{\frac 23} \frac{\partial}{\partial x} - \left( z^2 + \sin(x) \right) \frac{\partial}{\partial y}
\end{pmatrix}\\
&= \begin{pmatrix}
-\frac 29(x+y)^{-\frac 43}z -0 \\
...\\
....\\
\end{pmatrix} \neq \vec 0
\end{align*}
$\vec v$ besitzt somit kein Potential, da die hinreichende Bedingung nicht erfüllt ist.
\end{document}