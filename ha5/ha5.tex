\documentclass[10pt,a4paper,parskip=half]{scrartcl}
\usepackage[utf8]{inputenc}
\usepackage{amsmath}
\usepackage{amsfonts}
\usepackage{amssymb}
\usepackage{mathpazo}
\usepackage{stmaryrd} % Für den Widerspruchsblitz :D
\usepackage[a4paper,
left=3.0cm, right=3.0cm,
top=2.0cm, bottom=2.0cm]{geometry}
\usepackage{fullpage}
\usepackage[german]{babel}
\usepackage{enumerate}
\setlength{\unitlength}{1cm}
\newcommand{\N}{\mathbb{N}}
\newcommand{\A}{\mathcal{A}}
\newcommand{\R}{\mathbb{R}}
\parindent 0mm
\author{Tom}
\title{Analysis 2 - Hausaufgabe 5}

% new commands for vectors
\newcommand{\vectwo}[2]{\begin{pmatrix}#1\\#2\\\end {pmatrix}}
\newcommand{\vecthree}[3]{\begin{pmatrix}#1\\#2\\#3\\\end {pmatrix}}

\begin{document}
\begin{center}
\textsc{\Large{Analysis 2 - Hausaufgabe 4}} \\
\end{center}
\begin{tabbing}
Tom Nick \hspace{1.4cm}\= 342225\\
Tom Lehmann\> 340621\\
Maximilian Bachl\> 341455
\end{tabbing}
\section*{Aufgabe 1}
\begin{enumerate}[(a)]
\item Da $s = \frac g2 t^2$, gilt:
\begin{align*}
g &= \frac {2s}{t^2} = \frac {2 \cdot 44,5 \text{m}}{\left( 3,0\text{s}\right)^2} \approx 9,\overline{8} \frac {\text{m}}{\text{s}^2}
\end{align*}
\item Da wir den Fehlerschrankensatz anwenden sollen, benötigen wir zunächst die partiellen Ableitungen.
\begin{align*}
\frac {\partial g}{\partial s}(s,t) &= \frac 2{t^2}\\
\frac {\partial g}{\partial t}(s,t) &= \frac{-4s}{t^3}
\intertext{Als nächstes bestimmen wir $M_1$ sowie $M_2$:}
M_1 &= \sup_{\substack{s\in \left[s_0- \Delta s,s_0+\Delta s\right] \\ t \in \left[t_0 -\Delta t,t_0 + \Delta t\right]}}  \left| \frac{\partial g}{\partial s} (s,t) \right|\\
&= \sup_{\substack{s \in \left[44,4\text{m};44,6\text{m}\right] \\ t \in \left[2,9 \text{s};3,1\text{s}\right]}} \left| \frac 2{t^2} \right| \\
&=  \frac 2{2,9^2} = 0,238\\
M_2 &= \sup_{\substack{s\in \left[s_0- \Delta s,s_0+\Delta s\right] \\ t \in \left[t_0 -\Delta t,t_0 + \Delta t\right]}}  \left| \frac{\partial g}{\partial t} (s,t) \right|\\
&= \sup_{\substack{s \in \left[44,4\text{m};44,6\text{m}\right] \\ t \in \left[2,9 \text{s};3,1\text{s}\right]}} \left| \frac {4s}{t^3}\right| \\
&= \frac{-4 \cdot 44,6}{2,9^3} = 7,315
\intertext{Es gilt nun:}
\left| \Delta g \right| &= \left| \Delta g(s_0  + \Delta s, t_0 + \Delta t) - g(s_0,t_0) \right|\\
&\leq  M_1 |\Delta s| + M_2 |\Delta t | \\
&\leq 0,238 \cdot 0,1 + 7,315 \cdot 0,1 = 0,755
\end{align*}
Für die Gravitationskonstange  $g$ gilt also:
\begin{align*}
\left(9,\overline{8} -  0,755\right) \frac{\text{m}}{\text{s}^2} &\leq g \leq \left(9,\overline{8} + 0,755\right)  \frac{\text{m}}{\text{s}^2} \\
9,13\frac{\text{m}}{\text{s}^2} &\leq g \leq 10,64 \frac{\text{m}}{\text{s}^2}
\end{align*}
\end{enumerate}
\section*{Aufgabe 2}
Der näherungsweise Wert beträgt von  $e^{0.1}\cos(0.2)$ beträgt $1.083141$ (wir gehen von rad für den Winkel aus.

Wir definieren die Funktion $f(x,\psi) = e^x\cos(\psi)$. Dann ist
\begin{align*}
f'(x,\psi) &= 
\begin{pmatrix}
e^x\cos(\psi) & -e^x\sin(\psi)
\end{pmatrix} \\
f''(x,\psi) &= 
\begin{pmatrix}
e^x\cos(\psi) & -e^x\sin(\psi) \\
-e^x\sin(\psi) & -e^x\cos(\psi)
\end{pmatrix}
\end{align*}


Wir wählen für das Taylorpolynom 2. Ordnung den Entwicklungspunkt $\vec 0$.
\begin{align*} 
T_{\vec 0}(\vec x) &= 1 + \begin{pmatrix}1 & 0\end{pmatrix}\vectwo{x}{\psi} +\frac{1}{2} \begin{pmatrix}x & \psi\end{pmatrix}
\begin{pmatrix}
1 & 0 \\
0 & -1
\end{pmatrix}
\vectwo{x}{\psi} \\
&= 1 + x + \frac{1}{2}\begin{pmatrix}x & \psi\end{pmatrix}\vectwo{x}{-\psi} \\
&= 1 + x + \frac{1}{2}(x^2 - \phi^2)
\end{align*}

Somit ist $T_{\vec 0}(0.1, 0.2) = 1 + 0.1 + \frac{1}{2}(0.1^2 - 0.2^2) = 1.085$.

\end{document}