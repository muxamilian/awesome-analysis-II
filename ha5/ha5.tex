\documentclass[10pt,a4paper,parskip=half]{scrartcl}
\usepackage[utf8]{inputenc}
\usepackage{amsmath}
\usepackage{amsfonts}
\usepackage{amssymb}
\usepackage{mathpazo}
\usepackage{stmaryrd} % Für den Widerspruchsblitz :D
\usepackage[a4paper,
left=3.0cm, right=3.0cm,
top=2.0cm, bottom=2.0cm]{geometry}
\usepackage{fullpage}
\usepackage[german]{babel}
\usepackage{enumerate}
\setlength{\unitlength}{1cm}
\newcommand{\N}{\mathbb{N}}
\newcommand{\A}{\mathcal{A}}
\newcommand{\R}{\mathbb{R}}
\parindent 0mm
\author{Tom}
\title{Analysis 2 - Hausaufgabe 5}

% new commands for vectors
\newcommand{\vectwo}[2]{\begin{pmatrix}#1\\#2\\\end {pmatrix}}
\newcommand{\vecthree}[3]{\begin{pmatrix}#1\\#2\\#3\\\end {pmatrix}}

\begin{document}
\begin{center}
\textsc{\Large{Analysis 2 - Hausaufgabe 4}} \\
\end{center}
\begin{tabbing}
Tom Nick \hspace{1.4cm}\= 342225\\
Tom Lehmann\> 340621\\
Maximilian Bachl\> 341455
\end{tabbing}
\section*{Aufgabe 1}
\begin{enumerate}[(a)]
\item Da $s = \frac g2 t^2$, gilt:
\begin{align*}
g &= \frac {2s}{t^2} = \frac {2 \cdot 44,5 \text{m}}{\left( 3,0\text{s}\right)^2} \approx 9,\overline{8} \frac {\text{m}}{\text{s}^2}
\end{align*}
\item Da wir den Fehlerschrankensatz anwenden sollen, benötigen wir zunächst die partiellen Ableitungen.
\begin{align*}
\frac {\partial g}{\partial s}(s,t) &= \frac 2{t^2}\\
\frac {\partial g}{\partial t}(s,t) &= \frac{-4s}{t^3}
\intertext{Als nächstes bestimmen wir $M_1$ sowie $M_2$:}
M_1 &= \sup_{\substack{s\in \left[s_0- \Delta s,s_0+\Delta s\right] \\ t \in \left[t_0 -\Delta t,t_0 + \Delta t\right]}}  \left| \frac{\partial g}{\partial s} (s,t) \right|\\
&= \sup_{\substack{s \in \left[44,4\text{m};44,6\text{m}\right] \\ t \in \left[2,9 \text{s};3,1\text{s}\right]}} \left| \frac 2{t^2} \right| \\
&=  \frac 2{2,9^2} = 0,238\\
M_2 &= \sup_{\substack{s\in \left[s_0- \Delta s,s_0+\Delta s\right] \\ t \in \left[t_0 -\Delta t,t_0 + \Delta t\right]}}  \left| \frac{\partial g}{\partial t} (s,t) \right|\\
&= \sup_{\substack{s \in \left[44,4\text{m};44,6\text{m}\right] \\ t \in \left[2,9 \text{s};3,1\text{s}\right]}} \left| \frac {4s}{t^3}\right| \\
&= \frac{-4 \cdot 44,6}{2,9^3} = 7,315
\intertext{Es gilt nun:}
\left| \Delta g \right| &= \left| \Delta g(s_0  + \Delta s, t_0 + \Delta t) - g(s_0,t_0) \right|\\
&\leq  M_1 |\Delta s| + M_2 |\Delta t | \\
&\leq 0,238 \cdot 0,1 + 7,315 \cdot 0,1 = 0,755
\end{align*}
Für die Gravitationskonstange  $g$ gilt also:
\begin{align*}
\left(9,\overline{8} -  0,755\right) \frac{\text{m}}{\text{s}^2} &\leq g \leq \left(9,\overline{8} + 0,755\right)  \frac{\text{m}}{\text{s}^2} \\
9,13\frac{\text{m}}{\text{s}^2} &\leq g \leq 10,64 \frac{\text{m}}{\text{s}^2}
\end{align*}
\end{enumerate}
\section*{Aufgabe 2}
Der näherungsweise Wert beträgt von  $e^{0.1}\cos(0.2)$ beträgt $1.083141$ (wir gehen von rad für den Winkel aus).

Wir definieren die Funktion $f(x,\psi) = e^x\cos(\psi)$. Dann ist
\begin{align*}
f'(x,\psi) &= 
\begin{pmatrix}
e^x\cos(\psi) & -e^x\sin(\psi)
\end{pmatrix} \\
f''(x,\psi) &= 
\begin{pmatrix}
e^x\cos(\psi) & -e^x\sin(\psi) \\
-e^x\sin(\psi) & -e^x\cos(\psi)
\end{pmatrix}
\end{align*}


Wir wählen für das Taylorpolynom 2. Ordnung den Entwicklungspunkt $\vec 0$.
\begin{align*} 
T_{\vec 0}(\vec x) &= 1 + \begin{pmatrix}1 & 0\end{pmatrix}\vectwo{x}{\psi} +\frac{1}{2} \begin{pmatrix}x & \psi\end{pmatrix}
\begin{pmatrix}
1 & 0 \\
0 & -1
\end{pmatrix}
\vectwo{x}{\psi} \\
&= 1 + x + \frac{1}{2}\begin{pmatrix}x & \psi\end{pmatrix}\vectwo{x}{-\psi} \\
&= 1 + x + \frac{1}{2}(x^2 - \phi^2)
\end{align*}

Somit ist $T_{\vec 0}(0.1, 0.2) = 1 + 0.1 + \frac{1}{2}(0.1^2 - 0.2^2) = 1.085$. Die Abweichung dieser Abschätzung zur tatsächlichen Berechnung beträgt also $1.085 - 1.083141 = 0.001859$.
\section*{Aufgabe 3}
\begin{enumerate}[(i)]
	\item 	\(f\) ist eine zweimal partiell differenzierbare Funktion, da \(f\) eine Komposition aus zweimal partiell differenzierbaren Funktionen ist. Somit können wir, falls vorhanden, Extrema von $f$ finden. Jedes Extremum hat als notwendige Bedingung: $\nabla f(\vec x) = \vec 0$
	\[ \nabla f (\vec x) = 
	\begin{pmatrix}
		3x^2y - 3y \\
		x^3 - 3x + 2y 
	\end{pmatrix} =  
	\begin{pmatrix}
		y(3x^2 - 3) \\
		x^3 - 3x + 2y 
	\end{pmatrix}  \]
Nun müssen wir die Nullstellen finden.
Für \( y(3x^2 -3) = 0 \) gibt es 3 Lösungen:
\begin{enumerate}
	\item offenichtlich bei $y = 0$
	\item  und \(3x^2 - 3 = 0 \Leftrightarrow x^2 - 1 = 0 \Leftrightarrow x^2 = 1 \Rightarrow x = \pm 1\)
\end{enumerate}
Für \( x^3 - 3x + 2y  = 0 \) kann man keine direkten Nullstellen bestimmen.
Man kann nun in einer Matrix die möglichen Nullstellen von $\nabla f$ darstellen: \\
\begin{center}
\begin{math}
	\begin{array}{c|cc}
	 & x^3 - 3x + 2y  = 0  \\
	 \hline
	 y = 0 & (0,0),(\sqrt{3},0),(-\sqrt{3},0)\\
	 x = 1 & (1,1)\\
	 x = -1 & (-1,-1)
	\end{array}
\end{math}
\end{center}
Also sind die kritischen Punkte bei $(0,0), (1,-1),(-1,1),(\sqrt{3},0),(-\sqrt{3},0)$. Gesucht sind nun globale/lokale Extrema:
	\[H_f = 
	\begin{pmatrix}
		6xy & 3x^2 - 3 \\
		3x^2 - 3 & 2	
	\end{pmatrix} \]
	\begin{itemize}
		\item 	$(0,0)$
			\[H_f(0,0) = 
			\begin{pmatrix}
				0 & -3 \\
				-3 & 2		
			\end{pmatrix} \Rightarrow 
			\det \begin{pmatrix}
				0 & -3 \\
				-3 & 2		
			\end{pmatrix} = -9 \Rightarrow \text{ indefinit, Sattelpunkt}\]
		\item 	$(1,1)$
			\begin{align*}H_f(1,1) &= 
			\begin{pmatrix}
				6 & 0 \\
				0 & 2		
			\end{pmatrix} \Rightarrow 
			\det \begin{pmatrix}
				6 & 0 \\
				0 & 2		
			\end{pmatrix} = 12 \\
			&\Rightarrow \frac{\partial^2 f}{\partial x^2}(1,1) \text{ zu überprüfen. } \frac{\partial^2 f}{\partial x^2}(1,1) = 6 \Rightarrow \text{lokales Minimum}\\
			&f(1,1) = 0 \end{align*}
		\item  	$(-1,-1)$
			\begin{align*}H_f(-1,-1) &= 
			\begin{pmatrix}
				6 & 0 \\
				0 & 2		
			\end{pmatrix} \Rightarrow 
			\det \begin{pmatrix}
				6 & 0 \\
				0 & 2		
			\end{pmatrix} = 12 \\
			&\Rightarrow \frac{\partial^2 f}{\partial x^2}(-1,-1) \text{ zu überprüfen. } \frac{\partial^2 f}{\partial x^2}(-1,-1) = 6 \Rightarrow \text{lokales Minimum} \\
			&f(-1,-1) = 0 \end{align*}
		\item 	$(\sqrt{3},0)$
			\[H_f(\sqrt{3},0) = 
			\begin{pmatrix}
				0 & 6 \\
				6 & 2		
			\end{pmatrix} \Rightarrow 
			\det \begin{pmatrix}
				0 & 6 \\
				6 & 2	
			\end{pmatrix} = -36 \Rightarrow \text{indefinit, Sattelpunkt}\]
		\item  	$(-\sqrt{3},0)$
			\[H_f(-\sqrt{3},0) = 
			\begin{pmatrix}
				0 & 6 \\
				6 & 2		
			\end{pmatrix} \Rightarrow 
			\det \begin{pmatrix}
				0 & 6 \\
				6 & 2	
			\end{pmatrix} \approx -36 \Rightarrow \text{ indefinit, Sattelpunkt}\]
	\end{itemize}
	
	Bleibt noch zu überprüfen, ob die beiden gefunden lok. Minima auch glob. Minima sind.
	Das ist nicht so, da $\lim_{x \to \infty} f(x,1) = -\infty$ und $-\infty < f(1,1) = 0$ und $-\infty < f(-1,-1) = 0$.
	
	\item	
\(g\) ist eine zweimal partiell differenzierbare Funktion, da \(g\) eine Komposition aus zweimal partiell differenzierbaren Funktionen ist. Somit können wir, falls vorhanden, Extrema von $g$ finden. Jedes Extremum hat als notwendige Bedingung: $\nabla g(\vec x) = \vec 0$
	\[ \nabla f (\vec x) = 
	\begin{pmatrix}
		8x^3  +4xy^2\\
		 4x^2y + 4y^3
	\end{pmatrix} =  
	\begin{pmatrix}
		x(8x^2+4y^2) \\
		y(4x^2 +  4y^2)
	\end{pmatrix}  \]
Nun müssen wir die Nullstellen finden. Für $x(8x^2+4y^2)$:
\begin{enumerate}
\item offensichtlich bei $x=0$
\item und $8x^2+4y^2 = 0 \Leftrightarrow 2x^2 = -y^2 \Leftrightarrow y = \pm \sqrt{-2} x$
\end{enumerate}
Für $y(4x^2 +  4y^2)$:
\begin{enumerate}
\item offensichtlich bei $y=0$
\item und $4x^2 +  4y^2 = 0 \Leftrightarrow 4x^2 = 4y^2 \Leftrightarrow y = x$
\end{enumerate} 

Man kann nun in einer Matrix die möglichen Nullstellen von $\nabla g$ darstellen: \\
\begin{center}
\begin{math}
	\begin{array}{c|ccc}
	 & x = 0 & y = \sqrt{-2} x & y = - \sqrt{-2} x\\
	 \hline
	 y = 0 & (0,0) & (0,0) & (0,0) \\
	 y = x & (0,0) & \lightning & \lightning
	\end{array}
\end{math}
\end{center}
Also sind die kritischen Punkte bei $(0,0)$. Gesucht sind nun globale/lokale Extrema:
	\[H_g = 
	\begin{pmatrix}
		24x^2 + 4y^2 & 8xy \\
		8xy & 4x^2 + 12y^2
	\end{pmatrix} \]
	\begin{itemize}
		\item 	$(0,0)$
			\[H_g(0,0) = 
			\begin{pmatrix}
				0 & 0 \\
				0 & 0		
			\end{pmatrix} \Rightarrow 
			\det \begin{pmatrix}
				0 & 0 \\
				0 & 0		
			\end{pmatrix} = 0 \Rightarrow \text{ also können wir keine Aussage treffen.}\]\\
			$g(0,0) = 0$
			
			Da aber g eine Summe von positiven geraden Potenzen ist, die niemals kleiner als 0 werden können kann g niemals kleiner als 0 werden. Somit muss bei $(0,0)$ ein globales und lokales Minimum liegen.
	\end{itemize}
\end{enumerate}
\end{document}





