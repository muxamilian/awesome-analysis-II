\documentclass[10pt,a4paper]{article}
\usepackage[utf8]{inputenc}
\usepackage{amsmath}
\usepackage{amsfonts}
\usepackage{amssymb}
\usepackage{tikz}
\usetikzlibrary{patterns}
\usepackage[a4paper,
left=3.0cm, right=3.0cm,
top=2.0cm, bottom=2.0cm]{geometry}
\usepackage{fullpage}
\usepackage[german]{babel}
\usepackage{pst-all}
\usepackage{pstricks}
\usepackage{floatflt}
\setlength{\unitlength}{1cm}
\newcommand{\N}{\mathbb{N}}
\newcommand{\A}{\mathcal{A}}
\newcommand{\R}{\mathbb{R}}
\author{Tom}
\title{Analysis 2 - Hausaufgabe 1}
\begin{document}
\begin{center}
\Large{Analysis 2 - Hausaufgabe 1} \\
\end{center}
\begin{tabbing}
Tom Nick \hspace{1.4cm}\= 342225\\
Tom Lehmann\>  340621\\
Maximilian Bachl\> 123456\\\\ 
\end{tabbing}
\subsection*{Aufgabe 1.}
\subsubsection*{\textbf{(a)}}
\begin{floatingfigure}[r]{150pt}
\begin{tikzpicture}
	\begin{scope}[thick,font=\scriptsize]
%    % Axes:
%    % Are simply drawn using line with the `->` option to make them arrows:
%    % The main labels of the axes can be places using `node`s:
	 \draw [->] (-1,0) -- (2,0) node [right]  {$x$};
	 \draw [->] (0,-1) -- (0,2) node [above] {$y$};
	 \draw [-, densely dashed] (0,0) -- (2,2);
	 \def \bsp{(0,0) -- (2,2) -- (0,2)}
	 \pattern[pattern color=blue!60, pattern=north west lines]  \bsp;
	\end{scope}
\end{tikzpicture}
\end{floatingfigure}
\begin{align*}
\delta A &= \lbrace \left( x , y \right) \in \R ^2 : |x| = |y| , x \geq 0 \rbrace \\
\mathring{A} &= \lbrace \left( x,y \right) \in \R ^2 : |x| < |y|, x \geq 0 \rbrace
\end{align*}\\
Abgeschlossen? Nein, da  $\delta A \not \subseteq A$\\
Offen? Ja, da $\delta A \cap A = \emptyset$\\
Beschränkt? Nein\\
Kompakt? Nein, da A weder abgeschlossen, noch beschränkt ist.\\
\subsubsection*{\textbf{(b)}}
\begin{floatingfigure}[r]{140pt}
\begin{tikzpicture}
	\begin{scope}[thick,font=\scriptsize]
%    % Axes:
%    % Are simply drawn using line with the `->` option to make them arrows:
%    % The main labels of the axes can be places using `node`s:
	 \draw [->] (-2,0) -- (2.5,0) node [right]  {$x$};
	 \draw [->] (0,-1) -- (0,1.5) node [above] {$y$};
	 \draw [color = blue] (-1,-1) -- (-1,1.5) node [right] {$\frac {\pi}2$};
	 \draw [color = blue] (1,-1) -- (1,1.5) node [right] {$- \frac {\pi}2$};
	 \draw [color = blue] (2,-1) -- (2,1.5) node [right] {$\pi$};
	\end{scope}
\end{tikzpicture}
\end{floatingfigure}
\begin{align*}
\delta B &= \lbrace \left( x,y \right) \in \R ^2 : \sin \left( x \right) \cos \left( y \right) = 0 \rbrace \\
 \mathring{B} &= \emptyset
\end{align*}\\
Abgeschlossen? Ja, da $\delta B \subseteq B$.\\
Offen? Nein, da $\delta B \cap B = B \neq \emptyset$\\
Beschränkt? Nein, da $y \in \R$.\\
Kompakt? Nein, da $B$ zwar abgeschlossen, aber nicht beschränkt ist.\\
\subsubsection*{\textbf{(c)}}
\begin{floatingfigure}[r]{118pt}
\begin{tikzpicture}
	\begin{scope}[thick,font=\scriptsize]
%    % Axes:
%    % Are simply drawn using line with the `->` option to make them arrows:
%    % The main labels of the axes can be places using `node`s:
	 \draw [->] (1.5,1.5) -- (-1.5,-1.5) node [left] {$x$};
	 \draw [->] (-2,0) -- (2.5,0) node [right]  {$y$};
	 \draw [->] (0,-2) -- (0,2.5) node [above] {$z$};
	 \draw {(-0.71,-1.71) -- (0.29,-0.71) -- (-0.71,0.29)  -- (-1.71,-0.71) -- (-0.71,-1.71)};
	 \draw {(0.71,-0.29) -- (1.71,0.71) -- (0.71,1.71) -- (-0.29,0.71)};% -- (0.71,-0.29)};
	 \draw [-,dotted] (-0.29,0.71) -- (0.71,-0.29);

	 \draw {(-0.71,-1.71) -- (0.71,-0.29)};
	 \draw {(0.29,-0.71) -- (1.71,0.71)};
	 \draw {(-0.71,0.29) -- (0.71,1.71)};
	\end{scope} 
\end{tikzpicture}
\end{floatingfigure}
\begin{align*}
\delta C &= \lbrace \left( x,y,z \right) \in \R ^3 : |y|+|z| = 1, |x| = 1 \rbrace \\
\mathring{C} &= \lbrace \left( x,y,z \right) \in \R ^3 : |y| + |z| < 1 , |x| < 1 \rbrace 
\end{align*}\\
Abgeschlossen? Ja, da $\delta C \subseteq B$\\
Offen? Nein, da $\delta C \cap C \neq \emptyset$\\
Beschränkt? Ja, da $C \subseteq K_4(0,0,0)$\\
Kompakt? Ja, da $C$ sowohl abgeschlossen als auch beschränkt ist.\\\\

TODO: A-C
\newpage
\subsection*{Aufgabe 2.}
\subsubsection*{\textbf{(i)}}
\begin{tabbing}
Wenn A offen ist, gilt: $\delta A \cap A = \emptyset$\\
\hspace{0.6cm}\= Da $\delta A = \delta A^c$, muss gelten:
$\delta A^c \cap A \neq \emptyset \Rightarrow \delta A \subseteq A^c $\\
\> Also ist $A^c$ abgeschlossen, wenn $A$ offen ist.\\
Wenn A abgeschlossen ist, gilt: $\delta A \subseteq A$\\
\> Da $\delta A = \delta A^c$, muss gelten (da $A \not \subseteq A^c$):
$\delta A^c \cap A = \emptyset$\\
\> Also ist $A^C$ offen, wenn A abgeschlossen ist.
\end{tabbing}
Die Aussage ist somit wahr.
\subsubsection*{\textbf{(ii)}}
Kein Plan
\subsection*{Aufgabe 3.}
\subsubsection*{\textbf{(i)}}
\begin{align*}
&a_{k1}: \lim_{k \to \infty} \frac{ \arctan \left( k^2 \right) - \frac{\pi}2}{\frac 1k} = \text{!!!l'Hospital!!} \lim_{k \to \infty} \frac{\frac{1}{1+ k^4} \cdot 2k}{- \frac{1}{k^2}} = \lim_{k \to \infty} - \frac{2k^3}{1+k^4} = \lim_{k \to \infty} - \frac{\frac 2k}{\frac 1{k^4} + ^1} = 0\\
&a_{k2}: \lim_{k \to \infty} \frac 1{k^3} = 0\\
\\
&b_{k1}: \lim_{k \to \infty} \cos \left( \frac{\pi \cdot k}{2} \right)\\
&b_{k2}: \lim_{k \to \infty} \int_1^k \frac 1{t^2} dt = \lim_{k \to \inf} \left[ - \frac 1t \right]_1^k = \lim_{k \to \infty} -\frac 1t + 1 = 1 
\end{align*}
TEXT FEHLT
\subsubsection*{\textbf{(ii)}}
Kein Plan
%\subsubsection*{$y' = y^2 + \frac y x + \frac 1 {x^2}$}
%\begin{align*}
%y_1' &= (-2)^2 + \frac{-2}{\frac 12} + \frac 1{\frac 14} = 4 -4 + 4 = 4\\
%y_2' &= (-1)^2 - \frac 11 + \frac 1{1^2} = 1 - 1 + 1 = 1\\
%y_3' &= \left(-\frac 12\right)^2 + \frac {- \frac 12}{2} + \frac{1}{2^2} = \frac 14 - \frac 14 + \frac 14 = \frac 14
%\end{align*}
%\begin{floatingfigure}[l]{0.5\textwidth}
%\begin{center}
%\begin{tikzpicture}
%    \begin{scope}[thick,font=\scriptsize]
%    % Axes:
%    % Are simply drawn using line with the `->` option to make them arrows:
%    % The main labels of the axes can be places using `node`s:
%    \draw [->] (-1.5,0) -- (2.5,0) node [right]  {$x$};
%    \draw [->] (0,-2.5) -- (0,1.5) node [above] {$y$};
%
%    % Axes labels:
%    % Are drawn using small lines and labeled with `node`s. The placement can be set using options
%%    \iffalse% Single
%    % If you only want a single label per axis side:
%    \draw (2,-3pt) -- (2,3pt)   node [above] {$2$};
%    \draw (1,-3pt) -- (1,3pt)   node [above] {$1$};
%    \draw (-1,-3pt) -- (-1,3pt) node [above] {$-1$};
%    \draw (3pt,1) -- (-3pt,1)   node [left] {$1$};
%    \draw (3pt,-1) -- (-3pt,-1) node [left] {$-1$};
%    \draw (3pt,-2) -- (-3pt,-2)   node [left] {$-2$};
%%    \else% Multiple
%    % If you want labels at every unit step:
%%    \foreach \n in {-4,...,-1,1,2,...,4}{%
%%        \draw (\n,-3pt) -- (\n,3pt)   node [above] {$\n$};
%%        \draw (-3pt,\n) -- (3pt,\n)   node [right] {$\n i$};
%%    }
%%    \fi
%	\draw (0.9,-1.1) -- (1.1,-0.9);
%	\draw (1.9,-0.55) -- (2.1,-0.45);
%	\draw (0.45,-2.1) -- (0.55,-1.9);
%    \end{scope}
%    % The circle is drawn with `(x,y) circle (radius)`
%    % You can draw the outer border and fill the inner area differently.
%    % Here I use gray, semitransparent filling to not cover the axes below the circle
%    %\path [draw=none,fill=gray,semitransparent] (+1,-1) circle (3);
%    % Place the equation into the circle:
%    %\node [below right,darkgray] at (+1,-1) {$|z-1+i| \leq 3$};
%\end{tikzpicture}
%\end{center}
%\end{floatingfigure}
%Da $y: x \mapsto -\frac 1x$, gilt: $y' = \frac 1{x^2}$.\\
%Eingesetzt:
%\begin{align*}
%x^2 \cdot \frac 1 {x^2} &= x^2 \cdot \left( - \frac 1x \right) ^2 + x \cdot \left( - \frac 1x \right) +  1\\
%1 &= 1-1+1 \text{  w.A.}
%\end{align*}
%Demzufolge erfüllt $y: x \mapsto -\frac 1x$ die gegebene Differentialgleichung. Das Richtungsfeld in den gegebenen Punkten gibt die Steigung einer möglichen Lösungsfunktion (hier  $y: x \mapsto - \frac 1x$) in diesen Punkten an .
%%\setlength{\unitlength}{1cm}
%%\begin{pspicture}(0.5,3.5)(-3,-2)
%%
%%\put(2.6,-.1){$x$}
%%\put(-.1,1.7){$y$}
%%\put(0.2,0.9){$1$}
%%\put(0.9,0.2){$1$}
%%\linethickness{.075mm}
%%\psaxes[labels=none]{->}(0,0)(-1.5,-2.5)(2.5,1.5)
%%\linethickness{.3mm}
%%\psline(0.9,-1.1)(1.1,-0.9)
%%\psline(1.9,-0.55)(2.1,-0.45)
%%\psline(0.45,-2.1)(0.55,-1.9)
%%\end{pspicture}\\
%\subsection*{Aufgabe 2.}
%\subsubsection*{(a)}
%$ y' = -sin^2 (x)y + e^{\frac 12 (\sin(x)\cos(x)-x)}$
%\begin{align*}
%y_h &= c \cdot e^{\int -\sin ^2(x) dx} = c \cdot e ^{\frac 12 (\sin (x) \cos (x) -x )}\\
%y_p &= \int e^{-\frac 12 ( \sin (x) \cos (x) - x)} \cdot e^{\frac 12 \left( \sin (x) \cos (x) - x \right)} dx \cdot e ^{\frac 12 (\sin (x) \cos (x) -x )}\\
%&= \int {1}  dx \cdot e^{\frac 12 (\sin (x) \cos (x) -x )}\\
%&= x \cdot e^{\frac 12 (\sin (x) \cos (x) -x )}\\
%y(x) &= e^{\frac 12 (\sin (x) \cos (x) -x )} \left( c + x \right)
%\end{align*}
%Da $y(0) = 1$:\\
%
%$y(0) = 1 = e^0 \cdot \left( c + 0 \right) \Rightarrow c = 1$\\\\
%Also:\\
%
%$y(x) = e^{\frac 12 (\sin (x) \cos (x) -x )} \left( 1 + x \right)$
%
%\subsubsection*{(b)}
%$y' = y \cdot \left( \frac{-4x}{1+x^2} \right) + \left( \frac x{1+x^2} \right)$ mit $y(1) = 1$
%\begin{align*}
%y_h &= c \cdot e^{-2 \cdot \log \left( x^2 + 1 \right) }\\
%y_p &= c(x) \cdot e^{-2 \cdot \ln \left( x^2 + 1 \right) }\\
%&=  \int e^{2 \cdot \ln \left( x^2 + 1 \right) } \cdot \left( \frac x{1+x^2} \right) dx \cdot e^{-2 \cdot \ln \left( x^2 + 1 \right) }\\
%&= \int x \left(x^2 + 1 \right) dx \cdot e^{-2 \cdot \ln \left( x^2 + 1 \right) }\\
%&= \frac 14 x^4 + \frac 12 x^2 \cdot e^{-2 \cdot \ln \left( x^2 + 1 \right) }\\
%&=  x^2 \left( \frac 14 x^2 + \frac 12 \right) \cdot e^{-2 \cdot \ln \left( x^2 + 1 \right) }\\
%y(x) &= e^{-2 \cdot \ln \left( x^2 + 1 \right) } \cdot \left( c+ x^2 \left( \frac 14 x^2 + \frac 12 \right) \right)\\
%&= \left( x^2 + 1 \right)^{-2} \cdot \left( c+ x^2 \left( \frac 14 x^2 + \frac 12 \right) \right)
%\end{align*}
%Da $y(1) = 1$:\\
%
%$y(1) = 1 = 2^{-2} \cdot \left( c+ 1^2 \cdot \left( \frac 14 \cdot 1^2 + \frac 12 \right) \right) = c + \frac 14 \Rightarrow c = \frac 34$\\\\
%Also:\\
%
%$y(x) = \left( x^2 + 1 \right)^{-2} \cdot \left( \frac 34 + x^2 \left( \frac 14 x^2 + \frac 12 \right) \right)$
%
%\subsubsection*{(c)}
%$y' = sinc(x) \cdot y$
%Da es sich hier um eine homogene, lineare Differentialgleichung erster Ordnung handelt, gilt:
%\begin{align*}
%y(x) &= \tilde c \cdot e^{\int sinc \left( x \right) dx}
%\end{align*}
%Da gelten $y(0)=0$ gelten soll, muss gelten: $\tilde c = 0$. Also gilt:\\
%
%$y(x) = 0$
%\subsection*{Aufgabe 3.}
%\subsubsection*{(a)}
%\begin{align*}
%y_1 &= c_1 \cdot e^{-\int a \left( x \right) dx} + e^{- \int a \left( a \right) dx }  \cdot \int e^{\int a \left( x \right) dx} \cdot r_1 \left( x \right) dx\\
%&= e^{-\int a \left( x \right) dx} \left( c_1 + \int e^{\int a \left( x \right) dx} \cdot r_1 \left( x \right) dx \right)\\
%y_2 &= c_2 \cdot e^{- \int a \left( x \right) dx} + e^{- \int a \left( x \right) dx} \cdot \int e^{\int a \left( x \right) dx} \cdot r_2 \left( x \right) dx\\
%&= e^{- \int a \left( x \right) dx} \cdot \left( c_2 + \int e^{\int a \left( x \right) dx} \cdot r_2 \left( x \right) dx \right)\\
%&\\
%y_1 - y_2 &= e^{- \int a \left( x \right) dx}  \cdot \left( c_1 - c_2 + \int e^{\int a \left( x \right) dx} \cdot \left( r_1 \left( x \right) - r_2 \left( x \right) \right) dx \right)
%\end{align*}
\end{document}
