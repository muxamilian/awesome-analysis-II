\documentclass[10pt,a4paper,parskip=half]{scrartcl}
\usepackage[utf8]{inputenc}
\usepackage{amsmath}
\usepackage{amsfonts}
\usepackage{amssymb}
\usepackage{mathpazo}
\usepackage{tikz}
\usetikzlibrary{patterns}
\usepackage{stmaryrd} % Für den Widerspruchsblitz :D
\usepackage[left=1cm, right=1cm,
top=1cm, bottom=1cm]{geometry}
\usepackage{fullpage}
\usepackage[german]{babel}
\usepackage{enumerate}
\setlength{\unitlength}{1cm}
\newcommand{\N}{\mathbb{N}}
\newcommand{\A}{\mathcal{A}}
\newcommand{\R}{\mathbb{R}}
\parindent 0mm
\author{Tom}
\title{Analysis 2 - Hausaufgabe 12}

% new commands for vectors
\newcommand{\vectwo}[2]{\begin{pmatrix}#1\\#2\\\end {pmatrix}}
\newcommand{\vecthree}[3]{\begin{pmatrix}#1\\#2\\#3\\\end {pmatrix}}

\usepackage{listings}
\usepackage{courier}
 \lstset{
         basicstyle=\footnotesize\ttfamily, % Standardschrift
         %numbers=left,               % Ort der Zeilennummern
         numberstyle=\tiny,          % Stil der Zeilennummern
         %stepnumber=2,               % Abstand zwischen den Zeilennummern
         numbersep=5pt,              % Abstand der Nummern zum Text
         tabsize=2,                  % Groesse von Tabs
         extendedchars=true,         %
         breaklines=true,            % Zeilen werden Umgebrochen
         keywordstyle=\color{red},
      frame=b,         
 %        keywordstyle=[1]\textbf,    % Stil der Keywords
 %        keywordstyle=[2]\textbf,    %
 %        keywordstyle=[3]\textbf,    %
 %        keywordstyle=[4]\textbf,   \sqrt{\sqrt{}} %
         stringstyle=\color{white}\ttfamily, % Farbe der String
         showspaces=false,            % Leerzeichen anzeigen ?
         showtabs=false,              % Tabs anzeigen ?
         xleftmargin=17pt,
         framexleftmargin=17pt,
         framexrightmargin=5pt,
         framexbottommargin=4pt,
         %backgroundcolor=\color{lightgray},
         showstringspaces=false      % Leerzeichen in Strings anzeigen ?        
 }
 \usepackage{caption}
\DeclareCaptionFont{white}{\color{white}}
\DeclareCaptionFormat{listing}{\colorbox[cmyk]{0.43, 0.35, 0.35,0.01}{\parbox{\textwidth}{\hspace{15pt}#1#2#3}}}
\captionsetup[lstlisting]{format=listing,labelfont=white,textfont=white, singlelinecheck=false, margin=0pt, font={bf,footnotesize}}

\usepackage{color}
\usepackage{enumerate}



\begin{document}
\begin{center}
\textsc{\Large{Analysis 2 - Hausaufgabe 12}} \\
\end{center}
\begin{tabbing}
Tom Nick \hspace{1.4cm}\= 342225\\
Tom Lehmann\> 340621\\
Maximilian Bachl\> 341455
\end{tabbing}
\section*{1. Aufgabe}
\begin{itemize}
\item
Es gilt $\sqrt[k]{\left|a_k\right|} \le q < 1$ für alle $k$ ab einem gewissen $k_0$. 

Außerdem gilt trivialerweise $\sqrt[k]{\left|a_k\right|} \le q <1 \Leftrightarrow \left|a_k\right| \le q^k <1$. 

Weil $\sum_{k=0}^{\infty} \frac 1 {k^2}$ konvergiert und o.B.d.A $~q^k < \frac 1 {k^2}$ ab einem gewissen $k_1$ folgt, dass auch die Reihe $\sum_{k=0}^{\infty} a_k$ konvergiert.
\item
$$\sqrt[k]{|k^nx^k|} \le \sqrt {|x|} = \sqrt[k]{|k^n|}~x \le \sqrt {|x|}$$
Ab einem gewissen $k_0$ ist diese Gleichung erfüllt, da $\lim_{k\to\infty} \sqrt[k]{k^n} = 1$. Da aber $|x| < 1$ ist diese Gleichung erfüllt. Die $ \sqrt {|x|}$ steht hier für das $q$ aus dem vorherigen Beweis. Wir nehmen die Wurzel, da nur so die obige Formel gilt und $\sqrt {|x|}$ noch immer kleiner als $1$ ist. 

Nach dem Wurzelkriterium konvergiert diese Reihe also. Wenn $\left| x \right| > 0$ ist das Wurzelkriterium nicht mehr erfüllt und der Grenzwert der Folge bleibt größer als $1$. 
\end{itemize}
\section*{Aufgabe 2}
\begin{enumerate}[a)]
	\item 
	Ist nicht konvergent, da das notwendige Konvergenzkriterium nicht erfüllt ist:
		\begin{align*} 
			\lim_{n \to \infty} (-1)^n\frac{2n+7}{70n + 8} = \pm \frac 27 
		\end{align*}
         \item 
         Ist konvergent, da das Quontientenkriterium erfüllt ist:
            \begin{align*} 
               \lim_{n \to \infty} \left|\frac{\frac{1000^{n+1}}{(n+1)!}}{\frac{1000^n}{n!}}\right| = \lim_{n \to \infty} \left|\frac{\frac{1000^{n} + 1000}{(n+1) \cdot n!}}{\frac{1000^n}{n!}}\right| = \lim_{n \to \infty} \left|\frac{1000^{n} + 1000}{(n+1) 1000^n}\right| = 0 < 1
            \end{align*}
         Da offensichtlich gilt $\lim_{1 \to n} \frac{1000^n}{n!} = \lim_{1 \to n}  \left| \frac{1000^n}{n!} \right| $ ist die Reihe absolut konvergent.
         \item 
         Da offenlichtlich gilt $\lim_{1 \to n} \frac{n^5}{2^n+3^n} = 0$ ist die Reihe konvergent, da $\lim_{1 \to n} \frac{n^5}{2^n+3^n} = \lim_{1 \to n} \left| \frac{n^5}{2^n+3^n} \right|$ ist die Reihe auch absolut konvergent.


\end{enumerate}
\section*{Aufgabe 3}
\begin{enumerate}[a)]
	\item 
		\[ \sum_{k=1}^{\infty}(-1)^{k+1}\frac{(x-1)^k}{k} \]
\end{enumerate}
\end{document}

