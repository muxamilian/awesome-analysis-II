\documentclass[10pt,a4paper,parskip=half]{scrartcl}
\usepackage[utf8]{inputenc}
\usepackage{amsmath}
\usepackage{amsfonts}
\usepackage{amssymb}
\usepackage{mathpazo}
\usepackage{tikz}
\usetikzlibrary{patterns}
\usepackage{stmaryrd} % Für den Widerspruchsblitz :D
\usepackage[left=1cm, right=1cm,
top=1cm, bottom=1cm]{geometry}
\usepackage{fullpage}
\usepackage[german]{babel}
\usepackage{enumerate}
\setlength{\unitlength}{1cm}
\newcommand{\N}{\mathbb{N}}
\newcommand{\A}{\mathcal{A}}
\newcommand{\R}{\mathbb{R}}
\parindent 0mm
\author{Tom}
\title{Analysis 2 - Hausaufgabe 10}

% new commands for vectors
\newcommand{\vectwo}[2]{\begin{pmatrix}#1\\#2\\\end {pmatrix}}
\newcommand{\vecthree}[3]{\begin{pmatrix}#1\\#2\\#3\\\end {pmatrix}}

\usepackage{listings}
\usepackage{courier}
 \lstset{
         basicstyle=\footnotesize\ttfamily, % Standardschrift
         %numbers=left,               % Ort der Zeilennummern
         numberstyle=\tiny,          % Stil der Zeilennummern
         %stepnumber=2,               % Abstand zwischen den Zeilennummern
         numbersep=5pt,              % Abstand der Nummern zum Text
         tabsize=2,                  % Groesse von Tabs
         extendedchars=true,         %
         breaklines=true,            % Zeilen werden Umgebrochen
         keywordstyle=\color{red},
     	frame=b,         
 %        keywordstyle=[1]\textbf,    % Stil der Keywords
 %        keywordstyle=[2]\textbf,    %
 %        keywordstyle=[3]\textbf,    %
 %        keywordstyle=[4]\textbf,   \sqrt{\sqrt{}} %
         stringstyle=\color{white}\ttfamily, % Farbe der String
         showspaces=false,            % Leerzeichen anzeigen ?
         showtabs=false,              % Tabs anzeigen ?
         xleftmargin=17pt,
         framexleftmargin=17pt,
         framexrightmargin=5pt,
         framexbottommargin=4pt,
         %backgroundcolor=\color{lightgray},
         showstringspaces=false      % Leerzeichen in Strings anzeigen ?        
 }
 \usepackage{caption}
\DeclareCaptionFont{white}{\color{white}}
\DeclareCaptionFormat{listing}{\colorbox[cmyk]{0.43, 0.35, 0.35,0.01}{\parbox{\textwidth}{\hspace{15pt}#1#2#3}}}
\captionsetup[lstlisting]{format=listing,labelfont=white,textfont=white, singlelinecheck=false, margin=0pt, font={bf,footnotesize}}

\usepackage{color}
\usepackage{enumerate}



\begin{document}
\begin{center}
\textsc{\Large{Analysis 2 - Hausaufgabe 11}} \\
\end{center}
\begin{tabbing}
Tom Nick \hspace{1.4cm}\= 342225\\
Tom Lehmann\> 340621\\
Maximilian Bachl\> 341455
\end{tabbing}
\textbf{WARUM IGNORIERST DU MEINE GEOMETRY EINSTELLUNGEN, \LaTeX?}
\section*{Aufgabe 1}
\begin{enumerate}[(i)]
\item
\begin{align*}
\vec x&:~~ [0,2\pi[ \times [0,3] \to \mathbb{R}^3\\
\vec x(u,v) &= \vecthree{4\cos u}{v}{4\sin u} \\
\end{align*}
\item
\begin{align*}
\vec y&:~~ [0,2\pi[ \times [0,1] \to \mathbb{R}^3\\
\vec y(u,v) &= \vecthree{\sqrt v\cos u}{\sqrt v\sin u}{v} \\
\end{align*}
\end{enumerate}
\section*{Aufgabe 2}
Wir parametrisieren die Oberfläche:
\begin{align*}
\vec x&:~~ [0,2\pi[ \times [0,\pi] \to \mathbb{R}^3\\
\vec x(u,v) &= \vecthree{a\sin(u) \cos(v)}{b\cos(u)\sin(v)}{c \sin(u)} \\
\end{align*}
$\mathrm d \vec O$ ist somit:
\begin{align*}
\left(\frac{\partial \vec x}{\partial u} \times \frac{\partial \vec x}{\partial v}\right) \mathrm d u \mathrm d v &= \vecthree{a \cos u \cos v}{-b \sin u \sin v}{c \cos v} \times \vecthree{-a \sin u \sin v}{b \cos u \cos v}{0} \mathrm d u \mathrm d v\\
&= \vecthree{-bc \cos u \cos^2 v}{-ac \sin u \sin v \cos v}{ab \cos^2 u \cos^2 v - ab \sin^2 u \sin^2 v} \mathrm d u \mathrm d v\\
% Max Unfung
%\left|\frac{\partial \vec x}{\partial u} \times \frac{\partial \vec x}{\partial v}\right| \mathrm d u \mathrm d v &= \left|\vecthree{a \cos u \cos v}{-b \sin u \sin v}{c \cos v} \times \vecthree{-a \sin u \sin v}{b \cos u \cos v}{0}\right| \mathrm d u \mathrm d v\\
%&= \left|\vecthree{-bc \cos u \cos^2 v}{-ac \sin u \sin v \cos v}{ab \cos^2 u \cos^2 v - ab \sin^2 u \sin^2 v}\right| \mathrm d u \mathrm d v\\
%&= \sqrt{\left(-bc \cos u \cos^2 v\right)^2 + \left(-ac \sin u \sin v \cos v\right)^2 + \left(ab \cos^2 u \cos^2 v - ab \sin^2 u \sin^2 v\right)^2} \mathrm d u \mathrm d v
\end{align*}

Das gesuchte Integral ist somit:
\begin{align*}
&\int_0^{\pi} \int_0^{2\pi}\vec v(\vec x(u,v)) \cdot \begin{pmatrix}-bc \cos u \cos^2 v \\ -ac \sin u \sin v \cos v \\ab \cos^2 u \cos^2 v - ab \sin^2 u \sin^2 v \end{pmatrix} \mathrm d u \mathrm d v\\
 &= \int\limits_{0}^{\pi}\int\limits_{0}^{2\pi} \begin{pmatrix} -ac \sin u \sin v \cos v  \\ bc \cos u \cos^2 v \\ \left( ab \cos^2 u \cos^2 v - ab \sin^2 u \sin^2 v \right)^2\end{pmatrix}  \cdot \begin{pmatrix}-bc \cos u \cos^2 v \\ -ac \sin u \sin v \cos v \\ab \cos^2 u \cos^2 v - ab \sin^2 u \sin^2 v \end{pmatrix}\mathrm d u \mathrm d v 
 \intertext{TODO: ausmultiplizieren und integrieren. kb das inner bahn ohne wolfram zu machen. Der Fehler war das ihr ein skalares Oberflächenintegral berechnet habt (mit Betrag) hier aber ein Flussintegral berechnet werden soll/ wird. - Tom}
%&\int_0^{\pi} \int_0^{2\pi} f(\vec x(u,v)) \cdot \sqrt{\left(-bc \cos u \cos^2 v\right)^2 + \left(-ac \sin u \sin v \cos v\right)^2 + \left(ab \cos^2 u \cos^2 v - ab \sin^2 u \sin^2 v\right)^2} \mathrm d u \mathrm d v \\
%&= \int_0^{\pi} \int_0^{2\pi} \sqrt{\frac{(a\sin(u) \cos(v))^2}{a^4} + \frac{(b\cos(u)\sin(v))^2}{b^4} + \frac{(c \sin(u))^2}{c^4}} \\
%&\cdot \sqrt{\left(-bc \cos u \cos^2 v\right)^2 + \left(-ac \sin u \sin v \cos v\right)^2 + \left(ab \cos^2 u \cos^2 v - ab \sin^2 u \sin^2 v\right)^2} \mathrm d u \mathrm d v \\
%&= \int_0^{\pi} \int_0^{2\pi} \sqrt{\frac{\sin^2(u) \cos(v)^2}{a^2} + \frac{\cos^2(u)\sin^2(v)}{b^2} + \frac{\sin^2(u)}{c^2}} \\
%&\cdot \sqrt{\left(-bc \cos u \cos^2 v\right)^2 + \left(-ac \sin u \sin v \cos v\right)^2 + \left(ab \cos^2 u \cos^2 v - ab \sin^2 u \sin^2 v\right)^2} \mathrm d u \mathrm d v \\
%& = ...
\intertext{\textbf{Das kann doch nicht deren Ernst sein. -- Max}}
\intertext{\textbf{Isses auch nicht. -- Tom}}
\end{align*}
\section*{Aufgabe 3}
Die Parametrisierung für die Fläche $S$ ist wie folgt:
\begin{align*}
\vec y&:~~ [0,2\pi[ \times [0,1] \to \mathbb{R}^3\\
\vec y(u,v) &= \vecthree{\sqrt v\cos u}{\sqrt v\sin u}{v} \\
\end{align*}
$\mathrm d O$ ist somit:
\begin{align*}
\frac{\partial \vec y}{\partial u} \times \frac{\partial \vec y}{\partial v} ~ \mathrm d u \mathrm d v &= \vecthree{-\sqrt{v}\sin u}{\sqrt{v}\cos u}{0} \times \vecthree{\frac 1 2 v^{-\frac 1 2}\cos u}{\frac 1 2 v^{-\frac 1 2} \sin u}{1} \mathrm d u \mathrm d v\\
&= \vecthree{\sqrt{v}\cos u}{\sqrt{v}\sin u}{-\frac 1 2 \sin^2 u - \frac 1 2 \cos^2 u} \mathrm d u \mathrm d v\\
&= \vecthree{\sqrt{v}\cos u}{\sqrt{v}\sin u}{-\frac 1 2 (\sin^2 u + \cos^2 u)} \mathrm d u \mathrm d v\\
&= \vecthree{\sqrt{v}\cos u}{\sqrt{v}\sin u}{-\frac 1 2} \mathrm d u \mathrm d v\\
\end{align*}

Das gesuchte Integral ist somit:
\begin{align*}
\int_0^1 \int_0^{2\pi} \vec v(\vec y(u,v)) \cdot \vecthree{\sqrt{v}\cos u}{\sqrt{v}\sin u}{-\frac 1 2} \mathrm d u \mathrm d v &= \int_0^1 \int_0^{2\pi} \vecthree{\sqrt v\sin u}{-\sqrt v\cos u}{v^2} \cdot \vecthree{\sqrt{v}\cos u}{\sqrt{v}\sin u}{-\frac 1 2} \mathrm d u \mathrm d v \\
&= \int_0^1 \int_0^{2\pi} \left(v \sin u \cos u - v \sin u \cos u - \frac 1 2 v^2\right) \mathrm d u \mathrm d v \\
&= \int_0^1 \int_0^{2\pi} - \frac 1 2 v^2~ \mathrm d u \mathrm d v \\
&= \int_0^1 \left. - \frac 1 2 v^2 u~ \right|_0^{2\pi} \mathrm d v \\
&= \int_0^1 - \pi v^2 ~ \mathrm d v \\
&= \left. - \frac \pi 6 v^3 \right|_0^1 \\
&= -\frac \pi 6 \\
&\overset{\text{per Def. immer pos.}}{=} \frac \pi 6
\end{align*}
\section*{Aufgabe 4}
Die Winkel der Breiten betragen somit: $\theta_1 = 66.5^{\circ} = \frac{2\pi \cdot 66.5}{360} = 1.16$ und $\theta_2 = 113.5^{\circ} = \frac{2\pi \cdot 113.5}{360} = 1.98$

Die Parametrisierung ist also:
\begin{align*}
\vec z&:~~ [1.16,1.98] \times [0,2\pi[ \to \mathbb{R}^3\\
\vec z(u,v) &= 6378 \cdot \vecthree{\sin u \cos v}{\sin u \sin v}{\cos u} \\
\end{align*}
$\mathrm d O$ ist somit:
\begin{align*}
\left|\frac{\partial \vec y}{\partial u} \times \frac{\partial \vec y}{\partial v}\right| ~ \mathrm d u \mathrm d v&= 6378^2 \left|\vecthree{\cos u \cos v}{\cos u \sin v}{-\sin u} \times \vecthree{-\sin u \sin v}{\sin u \cos v}{0}\right| ~ \mathrm d u \mathrm d v\\
&= 6378^2 \left| \vecthree{\sin^2 u \cos v}{-\sin^2 u \sin v}{\cos u \sin u \cos^2 v + \sin u \cos u \sin^2 v} \right|~ \mathrm d u \mathrm d v\\
&= 6378^2 \sqrt{\left(\sin^2 u \cos v\right)^2 + \left(\sin^2 u \sin v\right)^2 + \left(\cos u \sin u \cos^2 v + \sin u \cos u \sin^2 v\right)^2}~ \mathrm d u \mathrm d v\\
&= 6378^2 \sqrt{\sin^4 u \cos^2 v + \sin^4 u \sin^2 v + \left(\cos u \sin u \cos^2 v + \sin u \cos u \sin^2 v\right)^2}~ \mathrm d u \mathrm d v\\
&= 6378^2 \sqrt{\sin^4 u \cos^2 v + \sin^4 u \sin^2 v + \left(\sin u \cos u\right)^2}~ \mathrm d u \mathrm d v\\
&= 6378^2 \sqrt{\sin^4 u \cos^2 v + \sin^4 u \sin^2 v + \sin^2 u \cos^2 u}~ \mathrm d u \mathrm d v\\
&= 6378^2 \sqrt{\sin^4 u\left(\cos^2 v + \sin^2 v\right) + \sin^2 u \cos^2 u}~ \mathrm d u \mathrm d v\\
&= 6378^2 \sqrt{\sin^4 u+ \sin^2 u \cos^2 u}~ \mathrm d u \mathrm d v\\
&= 6378^2 \sqrt{\sin^2 u \left(\sin^2 u +\cos^2 u\right)}~ \mathrm d u \mathrm d v\\
&= 6378^2 \sqrt{\sin^2 u}~ \mathrm d u \mathrm d v\\
&= 6378^2 \sin u~ \mathrm d u \mathrm d v\\
\end{align*}

Das Integral ist somit:
\begin{align*}
6379^2 \cdot \int_0^{2\pi} \int_{1.18}^{1.98} \sin u ~ \mathrm d u \mathrm d v &= 6379^2 \cdot \int_0^{2\pi} \left. -\cos u \right|_{1.18}^{1.98} ~ \mathrm d v\\
&= 6379^2 \left(-\cos 1.98 + \cos 1.18\right) \cdot \int_0^{2\pi}1~ \mathrm d v \\
&= 6379^2 \left(-\cos 1.98 + \cos 1.18\right) \cdot \left(\left. v \right|_0^{2\pi} \right) \\
&= 6379^2 \left(-\cos 1.98 + \cos 1.18\right) 2\pi \\
\end{align*}

\end{document}


