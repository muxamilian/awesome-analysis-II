\documentclass[10pt,a4paper,parskip=half]{scrartcl}
\usepackage[utf8]{inputenc}
\usepackage{amsmath}
\usepackage{amsfonts}
\usepackage{amssymb}
\usepackage{mathpazo}
\usepackage{tikz}
\usetikzlibrary{patterns}
\usepackage{stmaryrd} % Für den Widerspruchsblitz :D
\usepackage[a4paper,
left=3.0cm, right=3.0cm,
top=2.0cm, bottom=2.0cm]{geometry}
\usepackage{fullpage}
\usepackage[german]{babel}
\usepackage{enumerate}
\setlength{\unitlength}{1cm}
\newcommand{\N}{\mathbb{N}}
\newcommand{\A}{\mathcal{A}}
\newcommand{\R}{\mathbb{R}}
\parindent 0mm
\author{Tom}
\title{Analysis 2 - Hausaufgabe 10}

% new commands for vectors
\newcommand{\vectwo}[2]{\begin{pmatrix}#1\\#2\\\end {pmatrix}}
\newcommand{\vecthree}[3]{\begin{pmatrix}#1\\#2\\#3\\\end {pmatrix}}

\usepackage{listings}
\usepackage{courier}
 \lstset{
         basicstyle=\footnotesize\ttfamily, % Standardschrift
         %numbers=left,               % Ort der Zeilennummern
         numberstyle=\tiny,          % Stil der Zeilennummern
         %stepnumber=2,               % Abstand zwischen den Zeilennummern
         numbersep=5pt,              % Abstand der Nummern zum Text
         tabsize=2,                  % Groesse von Tabs
         extendedchars=true,         %
         breaklines=true,            % Zeilen werden Umgebrochen
         keywordstyle=\color{red},
     	frame=b,         
 %        keywordstyle=[1]\textbf,    % Stil der Keywords
 %        keywordstyle=[2]\textbf,    %
 %        keywordstyle=[3]\textbf,    %
 %        keywordstyle=[4]\textbf,   \sqrt{\sqrt{}} %
         stringstyle=\color{white}\ttfamily, % Farbe der String
         showspaces=false,            % Leerzeichen anzeigen ?
         showtabs=false,              % Tabs anzeigen ?
         xleftmargin=17pt,
         framexleftmargin=17pt,
         framexrightmargin=5pt,
         framexbottommargin=4pt,
         %backgroundcolor=\color{lightgray},
         showstringspaces=false      % Leerzeichen in Strings anzeigen ?        
 }
 \usepackage{caption}
\DeclareCaptionFont{white}{\color{white}}
\DeclareCaptionFormat{listing}{\colorbox[cmyk]{0.43, 0.35, 0.35,0.01}{\parbox{\textwidth}{\hspace{15pt}#1#2#3}}}
\captionsetup[lstlisting]{format=listing,labelfont=white,textfont=white, singlelinecheck=false, margin=0pt, font={bf,footnotesize}}

\usepackage{color}
\usepackage{enumerate}



\begin{document}
\begin{center}
\textsc{\Large{Analysis 2 - Hausaufgabe 11}} \\
\end{center}
\begin{tabbing}
Tom Nick \hspace{1.4cm}\= 342225\\
Tom Lehmann\> 340621\\
Maximilian Bachl\> 341455
\end{tabbing}
\section*{Aufgabe 1}
\begin{enumerate}[(i)]
\item
\begin{align*}
\vec x&:~~ [0,2\pi[ \times [0,3] \to \mathbb{R}^3\\
\vec x(u,v) &= \vecthree{4\cos u}{v}{4\sin u} \\
\end{align*}
\item
\begin{align*}
\vec y&:~~ [0,2\pi[ \times [0,1] \to \mathbb{R}^3\\
\vec y(u,v) &= \vecthree{\sqrt v\cos u}{\sqrt v\sin u}{v} \\
\end{align*}
\end{enumerate}
\section*{Aufgabe 2}
Wir parametrisieren die Oberfläche:
\begin{align*}
\vec x&:~~ [0,2\pi[ \times [0,\pi] \to \mathbb{R}^3\\
\vec x(u,v) &= \vecthree{a\sin(u) \cos(v)}{b\cos(u)\sin(v)}{c \sin(u)} \\
\end{align*}

$\mathrm d O$ ist somit:
\begin{align*}
\left|\frac{\partial \vec x}{\partial u} \times \frac{\partial \vec x}{\partial v}\right| \mathrm d u \mathrm d v &= \left|\vecthree{a \cos u \cos v}{-b \sin u \sin v}{c \cos v} \times \vecthree{-a \sin u \sin v}{b \cos u \cos v}{0}\right| \mathrm d u \mathrm d v\\
&= \left|\vecthree{-bc \cos u \cos^2 v}{-ac \sin u \sin v \cos v}{ab \cos^2 u \cos^2 v - ab \sin^2 u \sin^2 v}\right| \mathrm d u \mathrm d v\\
&= \sqrt{\left(-bc \cos u \cos^2 v\right)^2 + \left(-ac \sin u \sin v \cos v\right)^2 + \left(ab \cos^2 u \cos^2 v - ab \sin^2 u \sin^2 v\right)^2} \mathrm d u \mathrm d v
\end{align*}

Das gesuchte Integral ist somit:
\begin{align*}
&\int_0^{\pi} \int_0^{2\pi} f(\vec x(u,v)) \cdot \sqrt{\left(-bc \cos u \cos^2 v\right)^2 + \left(-ac \sin u \sin v \cos v\right)^2 + \left(ab \cos^2 u \cos^2 v - ab \sin^2 u \sin^2 v\right)^2} \mathrm d u \mathrm d v \\
&= \text{\textbf{Aufs Rechnen hatt ich dann nicht mehr so arg Lust... -- Max}}
\end{align*}
\section*{Aufgabe 3}
\section*{Aufgabe 4}
\end{document}


