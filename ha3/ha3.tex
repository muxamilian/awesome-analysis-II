\documentclass[10pt,a4paper,parskip=half]{scrartcl}
\usepackage[utf8]{inputenc}
\usepackage{amsmath}
\usepackage{amsfonts}
\usepackage{amssymb}
\usepackage{mathpazo}
\usepackage[a4paper,
left=3.0cm, right=3.0cm,
top=2.0cm, bottom=2.0cm]{geometry}
\usepackage{fullpage}
\usepackage[german]{babel}
\usepackage{enumerate}
\setlength{\unitlength}{1cm}
\newcommand{\N}{\mathbb{N}}
\newcommand{\A}{\mathcal{A}}
\newcommand{\R}{\mathbb{R}}
\parindent 0mm
\author{Tom}
\title{Analysis 2 - Hausaufgabe 3}

% new commands for vectors
\newcommand{\vectwo}[2]{\left(\begin {array} {c} #1 \\ #2 \\ \end {array} \right)}
\newcommand{\vecthree}[3]{\left(\begin {array} {c} #1 \\ #2 \\ #3\\ \end {array} \right) }

\begin{document}
\begin{center}
\textsc{\Large{Analysis 2 - Hausaufgabe 3}} \\
\end{center}
\begin{tabbing}
Tom Nick \hspace{1.4cm}\= 342225\\
Tom Lehmann\> 340621\\
Maximilian Bachl\> 341455
\end{tabbing}
\section{Aufgabe}
\begin{enumerate}[(i)]
\item
$f$ ist als Komposition differenzierbarer Funktionen auf $\R^2\setminus{\{(0,0)\}}$ differenzierbar und somit dort auch partiell differenzierbar. Bleibt dies noch für den Punkt $\vec x_0 = (0,0)$ zu zeigen.
\begin{align*}
\frac{\partial f}{\partial x} (0,0)&= \lim_{t \to 0} \frac{f(x+t,y)-f(0,0)}{t}\\
&= \lim_{t \to 0} \frac{\frac{t^20^2+0^8}{t^2+0^4} -0}{t} = 0
\end{align*}
\begin{align*}
\frac{\partial f}{\partial y} (0,0)&= \lim_{t \to 0} \frac{f(x,y+t)-f(0,0)}{t}\\
&= \lim_{t \to 0} \frac{\frac{0+t^8}{t^4} - 0}{t} \\
&= \lim_{t \to 0} \frac{t^4}{t} \\
&= \lim_{t \to 0} t^3 = 0
\end{align*}
Somit wurde gezeigt, dass f auch im Punkt (0,0) partiell differenzierbar ist.
% Unnötig
%\begin{align*}
%\frac{\partial f}{\partial x} &= \frac{2xy^2(x^2+y^4) - (x^2y^2+y^8)2x}{(x^2+y^4)^2}\\
%&= \frac{2x^3y^2+2xy^6 - 2x^3y^2-2xy^8}{(x^2+y^4)^2}\\
%&= \frac{2xy^6 -2xy^8}{(x^2+y^4)^2}\\
%\end{align*}
%\begin{align*}
%\frac{\partial f}{\partial y} &= \frac{(2yx^2 + 8y^7)(x^2+y^4)- (x^2y^2+y^8)(4y^3)}{(x^2+y^4)^2}\\
%&= \frac{2yx^4 + 8y^5x^4 +8x^2y^2 + 8y^6-4x^2y^5 - 4y^{11}}{(x^2+y^4)^2}\\
%\end{align*}

$g$ ist als Komposition differenzierbarer Funktionen auf $\R^2\setminus{\{(0,0)\}}$ differenzierbar und somit dort auch partiell differenzierbar. Bleibt dies noch für den Punkt $\vec x_0 = (0,1)$ zu zeigen.
\begin{align*}
\frac{\partial g}{\partial x} (0,1)&= \lim_{t \to 0} \frac{g(x+t,y)-g(0,1)}{t}\\
&= \lim_{t \to 0} \frac{\frac{t^40^2+t^30^3}{(t^2+0^2)^3} - 0}{t}\\
&= \lim_{t \to 0} \frac{0}{t} = 0
\end{align*}
\begin{align*}
\frac{\partial g}{\partial y} (0,1)&= \lim_{t \to 0} \frac{g(x,y+t)-g(0,1)}{t}\\
&= \lim_{t \to 0} \frac{\frac{0}{(0^2+t^2)^3} - 0}{t}\\
&= \lim_{t \to 0} \frac{0}{t} = 0
\end{align*}
\newpage
\item
Da f überall auf $\mathbb R^2\setminus{\{(0,0)\}}$ eine Komposition differenzierbarer Funktionen ist, ist sie dort auch differenzierbar. Bleibt dies noch für den Punkt (0,0) zu zeigen.

$$\text{Rest} = f(x,y) - f(0,0) - f'(0,0) \Delta \vec x = \frac{x^2y^2+y^8}{x^2 + y^4} - 0 - (0,0) \vectwo{x}{y} = \frac{x^2y^2+y^8}{x^2 + y^4}$$

% Kleines Problem mit den Vektoren ;)
Es ist Z.z. $$\lim_{\vectwo{x}{y} \to \vectwo{0}{0}} \left|\frac{\frac{x^2y^2+y^8}{x^2 + y^4}}{\left|\vectwo{x}{y}\right|}\right| = 0$$

\begin{align*}
0 &\le \lim_{\Delta \vec x \to \vec 0} \left|\frac{\frac{x^2y^2+y^8}{x^2 + y^4}}{\left|\vectwo{x}{y}\right|}\right|\\
&= \lim_{\Delta \vec x \to \vec 0} \left|\frac{\frac{x^2y^2+y^8}{x^2 + y^4}}{\sqrt{x^2+y^2}}\right| \\
&= \lim_{\Delta \vec x \to \vec 0} \left|\frac{x^2y^2+y^8}{(x^2 + y^4)\sqrt{x^2+y^2}}\right| \\
&\le \lim_{\Delta \vec x \to \vec 0} \left|\frac{x^2y^2+y^8}{y^4\sqrt{y^2}}\right| \\
&= \lim_{\Delta \vec x \to \vec 0} \left|\frac{x^2y^2+y^8}{y^5}\right| \\
&= \lim_{\Delta \vec x \to \vec 0} \left|\frac{x^2y+y^7}{y^4}\right| \text{ mit } (y^2 -1)^2 \ge 0 \Leftrightarrow y^4 - 2y^2 + 1 \ge 0 \Leftrightarrow y^4 \ge 2y^2 -1\\
&\le \lim_{\Delta \vec x \to \vec 0} \left|\frac{x^2y+y^7}{2y^2 -1}\right| \\
&= \frac{0}{-1} = 0
\end{align*}

Da $g$ -- wie in der Analysis-Aufgabe der letzten Woche bewiesen -- nicht stetig ist, kann es auch nicht differenzierbar sein.
\newpage
\section{Aufgabe}
Da h überall auf $\mathbb R^2\setminus{\{(0,0)\}}$ eine Komposition differenzierbarer Funktionen ist, ist sie dort auch differenzierbar. Bleibt dies noch für den Punkt (0,0) zu zeigen.

$\text{Rest} = h(x,y) - h(0,0) - h'(0,0) \Delta \vec x = \frac{\text{sin}^4(x)\text{sin}^4(y)}{x^2 + y^2} - 0 - (0,0) \vectwo{x}{y} = \frac{\text{sin}^4(x)\text{sin}^4(y)}{x^2 + y^2}$

Z.z. $\lim_{\vectwo{x}{y} \to \vectwo{0}{0}} \left|\frac{\frac{\text{sin}^4(x)\text{sin}^4(y)}{x^2 + y^2}}{\left|\vectwo{x}{y}\right|}\right| = 0$

\begin{align*}
0 &\le\lim_{\Delta \vec x \to \vec 0} \left|\frac{\frac{\text{sin}^4(x)\text{sin}^4(y)}{x^2 + y^2}}{\left|\vectwo{x}{y}\right|}\right| \\
&= \lim_{\Delta \vec x \to \vec 0} \left|\frac{\frac{\text{sin}^4(x)\text{sin}^4(y)}{x^2 + y^2}}{\sqrt{x^2+y^2}}\right| \\
&= \lim_{\Delta \vec x \to \vec 0} \left|\frac{\text{sin}^4(x)\text{sin}^4(y)}{(x^2 + y^2)\sqrt{x^2+y^2}}\right| \\
&= \lim_{\Delta \vec x \to \vec 0} \left|\frac{\text{sin}^4(x)\text{sin}^4(y)}{(x^2 + y^2)^{\frac{3}{2}}}\right| \\
&\le \lim_{\Delta \vec x \to \vec 0} \left|\frac{\text{sin}^4(x)\text{sin}^4(y)}{(2xy)^{\frac{3}{2}}}\right| \\
&\le \lim_{\Delta \vec x \to \vec 0} \left|\frac{x^4y^4}{(2xy)^{\frac{3}{2}}}\right| \\
&= \lim_{\Delta \vec x \to \vec 0} \left|\frac{(xy)^4}{(2xy)^{\frac{3}{2}}}\right| \\
&\le \lim_{\Delta \vec x \to \vec 0} \left|\frac{(2xy)^4}{(2xy)^{\frac{3}{2}}}\right| \\
&= \lim_{\Delta \vec x \to \vec 0} \left|(2xy)^{\frac{5}{2}}\right| \\
&= 0
\end{align*}
Somit ist $h$ total differenzierbar.

\[
h'(x,y): \mathbb{R}^2\setminus{\{(0,0)\}} \to \mathbb{R}, 
\]
\[
(x,y) \mapsto \left( \begin{array}{cc} 
\frac{4\cos(x)\sin^3(x)\sin^4(y)(x^2 + y^2) - \sin^4(x)\sin^4(y)2x}{(x^2 + y^2)^2}, &
\frac{4\cos(y)\sin^3(y)\sin^4(x)(x^2 + y^2) - \sin^4(x)\sin^4(y)2y}{(x^2 + y^2)^2}
\end{array}\right)
\]

Der Gradient am gegebenen Punkt ist gleichbedeutend mit dem der Richtung des steilsten Anstiegs dort. \(\nabla h(\frac{\pi}{2},\frac{\pi}{2}) = \vectwo{\frac{4\pi}{\pi^4}}{\frac{4\pi}{\pi^4}}\) BITTE NACHRECHNEN

\begin{align*}
\frac{\partial h}{\partial \vec v} (P) &= < \nabla h \vectwo{\frac{\pi}{2}}{\frac{\pi}{2}}, \vectwo{v_1}{v_2} >\\
 &= < \vectwo{\frac{4\cos(\frac{\pi}{2})\sin^3(\frac{\pi}{2})\sin^4(\frac{\pi}{2})(\frac{\pi}{2}^2 + \frac{\pi}{2}^2) - \sin^4(\frac{\pi}{2})\sin^4(\frac{\pi}{2})2\frac{\pi}{2}}{(\frac{\pi}{2}^2 + \frac{\pi}{2}^2)^2}}{\frac{4\cos(\frac{\pi}{2})\sin^3(\frac{\pi}{2})\sin^4(\frac{\pi}{2})(\frac{\pi}{2}^2 + \frac{\pi}{2}^2) - \sin^4(\frac{\pi}{2})\sin^4(\frac{\pi}{2})2\frac{\pi}{2}}{(\frac{\pi}{2}^2 + \frac{\pi}{2}^2)^2}}, \vectwo{v_1}{v_2} >\\
 &= < \vectwo{\frac{-4}{\pi^3}}{\frac{-4}{\pi^3}}, \vectwo{v_1}{v_2} > \\
 &= \frac{v_1}{\pi^3} + \frac{v_2}{\pi^3} \\ \\
 0 &= \frac{0}{\pi^3} + \frac{0}{\pi^3}
\end{align*}
$\vec v = \vectwo{0}{0}$ ist also ein Vektor, bei dem die Richtungsableitung im Punkt P den Wert 0 annimmt.

\section{Aufgabe}
$h$ ist differenzierbar, da auf dem definierten Definitionsbereich die Komposition der Funktionen differenzierbar ist. Die Ableitungsmatrix ist:
$$h' =\begin{pmatrix}
\frac{2x}{x^2 + y^2 + z^2} & \frac{2y}{x^2 + y^2 + z^2} & \frac{2z}{x^2 + y^2 + z^2} \\
2x - yz & -xz & -xy \\
yze^{xyz}  & xze^{xyz} & xye^{xyz} 
\end{pmatrix}  $$


\end{enumerate}
\end{document}