\documentclass[10pt,a4paper,parskip=half]{scrartcl}
\usepackage[utf8]{inputenc}
\usepackage{amsmath}
\usepackage{amsfonts}
\usepackage{amssymb}
\usepackage{tikz}
\usepackage{mathpazo}
\usetikzlibrary{patterns}
\usepackage[a4paper,
left=3.0cm, right=3.0cm,
top=2.0cm, bottom=2.0cm]{geometry}
\usepackage{fullpage}
\usepackage[german]{babel}
\usepackage{pst-all}
\usepackage{pstricks}
\usepackage{floatflt}
\usepackage{listings}
\usepackage{courier}
 \lstset{
         basicstyle=\footnotesize\ttfamily, % Standardschrift
         %numbers=left, % Ort der Zeilennummern
         numberstyle=\tiny, % Stil der Zeilennummern
         %stepnumber=2, % Abstand zwischen den Zeilennummern
         numbersep=5pt, % Abstand der Nummern zum Text
         tabsize=2, % Groesse von Tabs
         extendedchars=true, %
         breaklines=true, % Zeilen werden Umgebrochen
         keywordstyle=\color{red},
     frame=b,
 % keywordstyle=[1]\textbf, % Stil der Keywords
 % keywordstyle=[2]\textbf, %
 % keywordstyle=[3]\textbf, %
 % keywordstyle=[4]\textbf, \sqrt{\sqrt{}} %
         stringstyle=\color{white}\ttfamily, % Farbe der String
         showspaces=false, % Leerzeichen anzeigen ?
         showtabs=false, % Tabs anzeigen ?
         xleftmargin=17pt,
         framexleftmargin=17pt,
         framexrightmargin=5pt,
         framexbottommargin=4pt,
         %backgroundcolor=\color{lightgray},
         showstringspaces=false % Leerzeichen in Strings anzeigen ?
 }
 \usepackage{caption}
\DeclareCaptionFont{white}{\color{white}}
\DeclareCaptionFormat{listing}{\colorbox[cmyk]{0.43, 0.35, 0.35,0.01}{\parbox{\textwidth}{\hspace{15pt}#1#2#3}}}
\captionsetup[lstlisting]{format=listing,labelfont=white,textfont=white, \text{sin}glelinecheck=false, margin=0pt, font={bf,footnotesize}}

\usepackage{color}
\usepackage{enumerate}
\setlength{\unitlength}{1cm}
\newcommand{\N}{\mathbb{N}}
\newcommand{\A}{\mathcal{A}}
\newcommand{\R}{\mathbb{R}}
\parindent 0mm
\author{Tom}
\title{Analysis 2 - Hausaufgabe 3}

% new commands for vectors
\newcommand{\vectwo}[2]{\left(\begin {array} {c} #1 \\ #2 \\ \end {array} \right)}
\newcommand{\vecthree}[3]{\left(\begin {array} {c} #1 \\ #2 \\ #3\\ \end {array} \right) }

\begin{document}
\begin{center}
\textsc{\Large{Analysis 2 - Hausaufgabe 3}} \\
\end{center}
\begin{tabbing}
Tom Nick \hspace{1.4cm}\= 342225\\
Tom Lehmann\> 340621\\
Maximilian Bachl\> 341455
\end{tabbing}
\section{Aufgabe}
\begin{enumerate}[(i)]
\item
$f$ ist als Komposition differenzierbarer Funktionen auf $\R^2\setminus{\{(0,0)\}}$ differenzierbar und somit dort auch partiell differenzierbar. Bleibt dies noch für den Punkt $\vec x_0 = (0,0)$ zu zeigen.
\begin{align*}
\frac{\partial f}{\partial x} (0,0)&= \lim_{t \to 0} \frac{f(x+t,y)-f(0,0)}{t}\\
&= \lim_{t \to 0} \frac{\frac{t^20^2+0^8}{t^2+0^4} -0}{t} = 0
\end{align*}
\begin{align*}
\frac{\partial f}{\partial y} (0,0)&= \lim_{t \to 0} \frac{f(x,y+t)-f(0,0)}{t}\\
&= \lim_{t \to 0} \frac{\frac{0+t^8}{t^4} - 0}{t} \\
&= \lim_{t \to 0} \frac{t^4}{t} \\
&= \lim_{t \to 0} t^3 = 0
\end{align*}
Somit wurde gezeigt, dass f auch im Punkt (0,0) partiell differenzierbar ist.
% Unnötig
%\begin{align*}
%\frac{\partial f}{\partial x} &= \frac{2xy^2(x^2+y^4) - (x^2y^2+y^8)2x}{(x^2+y^4)^2}\\
%&= \frac{2x^3y^2+2xy^6 - 2x^3y^2-2xy^8}{(x^2+y^4)^2}\\
%&= \frac{2xy^6 -2xy^8}{(x^2+y^4)^2}\\
%\end{align*}
%\begin{align*}
%\frac{\partial f}{\partial y} &= \frac{(2yx^2 + 8y^7)(x^2+y^4)- (x^2y^2+y^8)(4y^3)}{(x^2+y^4)^2}\\
%&= \frac{2yx^4 + 8y^5x^4 +8x^2y^2 + 8y^6-4x^2y^5 - 4y^{11}}{(x^2+y^4)^2}\\
%\end{align*}

$g$ ist als Komposition differenzierbarer Funktionen auf $\R^2\setminus{\{(0,0)\}}$ differenzierbar und somit dort auch partiell differenzierbar. Bleibt dies noch für den Punkt $\vec x_0 = (0,1)$ zu zeigen.
\begin{align*}
\frac{\partial g}{\partial x} (0,1)&= \lim_{t \to 0} \frac{g(x+t,y)-g(0,1)}{t}\\
&= \lim_{t \to 0} \frac{\frac{t^40^2+t^30^3}{(t^2+0^2)^3} - 0}{t}\\
&= \lim_{t \to 0} \frac{0}{t} = 0
\end{align*}
\begin{align*}
\frac{\partial g}{\partial y} (0,1)&= \lim_{t \to 0} \frac{g(x,y+t)-g(0,1)}{t}\\
&= \lim_{t \to 0} \frac{\frac{0}{(0^2+t^2)^3} - 0}{t}\\
&= \lim_{t \to 0} \frac{0}{t} = 0
\end{align*}

\item
Da f überall auf $\mathbb R^2\setminus{\{(0,0)\}}$ eine Komposition differenzierbarer Funktionen ist, ist sie dort auch differenzierbar. Bleibt dies noch für den Punkt (0,0) zu zeigen.

$\text{Rest} = f(x,y) - f(0,0) - f'(0,0) \Delta \vec x = \frac{x^2y^2+y^8}{x^2 + y^4} - 0 - (0,0) \vectwo{x}{y} = \frac{x^2y^2+y^8}{x^2 + y^4}$

% Kleines Problem mit den Vektoren ;)
Z.z. $\lim_{\vectwo{x}{y} \to \vectwo{0}{0}} \left|\frac{\frac{x^2y^2+y^8}{x^2 + y^4}}{\left|\vectwo{x}{y}\right|}\right| = 0$

\begin{align*}
0 &\le \lim_{\Delta \vec x \to \vec 0} \left|\frac{\frac{x^2y^2+y^8}{x^2 + y^4}}{\left|\vectwo{x}{y}\right|}\right|\\
&= \lim_{\Delta \vec x \to \vec 0} \left|\frac{\frac{x^2y^2+y^8}{x^2 + y^4}}{\sqrt{x^2+y^2}}\right| \\
&\le \lim_{\Delta \vec x \to \vec 0} \left|\frac{\frac{x^2y^2+y^8}{2xy^2}}{\sqrt{x^2+y^2}}\right| \\
&\le \lim_{\Delta \vec x \to \vec 0} \left|\frac{\frac{x^2y^2+y^8}{xy^2}}{\sqrt{x^2+y^2}}\right| \\
&= \lim_{\Delta \vec x \to \vec 0} \left|\frac{\frac{x^2+y^6}{x}}{\sqrt{x^2+y^2}}\right| \\
&= \lim_{\Delta \vec x \to \vec 0} \left|\frac{x+\frac{y^6}{x}}{\sqrt{x^2+y^2}}\right| \\
& \text{Ich geb vorerst auf -- Max}
\end{align*}

Da $g$ -- wie in der Analysis-Aufgabe der letzten Woche bewiesen -- nicht stetig ist, kann es auch nicht differenzierbar sein.
\section{Aufgabe}
Da h überall auf $\mathbb R^2\setminus{\{(0,0)\}}$ eine Komposition differenzierbarer Funktionen ist, ist sie dort auch differenzierbar. Bleibt dies noch für den Punkt (0,0) zu zeigen.

$\text{Rest} = h(x,y) - h(0,0) - h'(0,0) \Delta \vec x = \frac{\text{sin}^4(x)\text{sin}^4(y)}{x^2 + y^2} - 0 - (0,0) \vectwo{x}{y} = \frac{\text{sin}^4(x)\text{sin}^4(y)}{x^2 + y^2}$

Z.z. $\lim_{\vectwo{x}{y} \to \vectwo{0}{0}} \left|\frac{\frac{\text{sin}^4(x)\text{sin}^4(y)}{x^2 + y^2}}{\left|\vectwo{x}{y}\right|}\right| = 0$

\begin{align*}
0 &\le\lim_{\Delta \vec x \to \vec 0} \left|\frac{\frac{\text{sin}^4(x)\text{sin}^4(y)}{x^2 + y^2}}{\left|\vectwo{x}{y}\right|}\right| \\
&= \lim_{\Delta \vec x \to \vec 0} \left|\frac{\frac{\text{sin}^4(x)\text{sin}^4(y)}{x^2 + y^2}}{\sqrt{x^2+y^2}}\right| \\
&= \lim_{\Delta \vec x \to \vec 0} \left|\frac{\text{sin}^4(x)\text{sin}^4(y)}{(x^2 + y^2)\sqrt{x^2+y^2}}\right| \\
&= \lim_{\Delta \vec x \to \vec 0} \left|\frac{\text{sin}^4(x)\text{sin}^4(y)}{(x^2 + y^2)^{\frac{3}{2}}}\right| \\
&\le \lim_{\Delta \vec x \to \vec 0} \left|\frac{\text{sin}^4(x)\text{sin}^4(y)}{(2xy)^{\frac{3}{2}}}\right| \\
\end{align*}
Somit ist $h$ total differenzierbar.

\section{Aufgabe}
$h$ ist differenzierbar, da auf dem definierten Definitionsbereich die Komposition der Funktionen differenzierbar ist. Die Ableitungsmatrix ist:
$$h' =\begin{pmatrix}
\frac{2x}{x^2 + y^2 + z^2} & \frac{2y}{x^2 + y^2 + z^2} & \frac{2z}{x^2 + y^2 + z^2} \\
2x - yz & -xz & -xy \\
yze^{xyz}  & xze^{xyz} & xye^{xyz} 
\end{pmatrix}  $$


\end{enumerate}
\end{document}