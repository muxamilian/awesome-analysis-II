\documentclass[10pt,a4paper,parskip=half]{scrartcl}
\usepackage[utf8]{inputenc}
\usepackage{amsmath}
\usepackage{amsfonts}
\usepackage{amssymb}
\usepackage{mathpazo}
\usepackage{tikz}
\usetikzlibrary{patterns}
\usepackage{stmaryrd} % Für den Widerspruchsblitz :D
\usepackage[a4paper,
left=3.0cm, right=3.0cm,
top=2.0cm, bottom=2.0cm]{geometry}
\usepackage{fullpage}
\usepackage[german]{babel}
\usepackage{enumerate}
\setlength{\unitlength}{1cm}
\newcommand{\N}{\mathbb{N}}
\newcommand{\A}{\mathcal{A}}
\newcommand{\R}{\mathbb{R}}
\parindent 0mm
\author{Tom}
\title{Analysis 2 - Hausaufgabe 10}

% new commands for vectors
\newcommand{\vectwo}[2]{\begin{pmatrix}#1\\#2\\\end {pmatrix}}
\newcommand{\vecthree}[3]{\begin{pmatrix}#1\\#2\\#3\\\end {pmatrix}}

\usepackage{listings}
\usepackage{courier}
 \lstset{
         basicstyle=\footnotesize\ttfamily, % Standardschrift
         %numbers=left,               % Ort der Zeilennummern
         numberstyle=\tiny,          % Stil der Zeilennummern
         %stepnumber=2,               % Abstand zwischen den Zeilennummern
         numbersep=5pt,              % Abstand der Nummern zum Text
         tabsize=2,                  % Groesse von Tabs
         extendedchars=true,         %
         breaklines=true,            % Zeilen werden Umgebrochen
         keywordstyle=\color{red},
    		frame=b,         
 %        keywordstyle=[1]\textbf,    % Stil der Keywords
 %        keywordstyle=[2]\textbf,    %
 %        keywordstyle=[3]\textbf,    %
 %        keywordstyle=[4]\textbf,   \sqrt{\sqrt{}} %
         stringstyle=\color{white}\ttfamily, % Farbe der String
         showspaces=false,           % Leerzeichen anzeigen ?
         showtabs=false,             % Tabs anzeigen ?
         xleftmargin=17pt,
         framexleftmargin=17pt,
         framexrightmargin=5pt,
         framexbottommargin=4pt,
         %backgroundcolor=\color{lightgray},
         showstringspaces=false      % Leerzeichen in Strings anzeigen ?        
 }
 \usepackage{caption}
\DeclareCaptionFont{white}{\color{white}}
\DeclareCaptionFormat{listing}{\colorbox[cmyk]{0.43, 0.35, 0.35,0.01}{\parbox{\textwidth}{\hspace{15pt}#1#2#3}}}
\captionsetup[lstlisting]{format=listing,labelfont=white,textfont=white, singlelinecheck=false, margin=0pt, font={bf,footnotesize}}

\usepackage{color}
\usepackage{enumerate}



\begin{document}
\begin{center}
\textsc{\Large{Analysis 2 - Hausaufgabe 10}} \\
\end{center}
\begin{tabbing}
Tom Nick \hspace{1.4cm}\= 342225\\
Tom Lehmann\> 340621\\
Maximilian Bachl\> 341455
\end{tabbing}
\section*{Aufgabe 1}
\begin{enumerate}[a)]
   \item 
            \begin{align*}
               \int\limits^{1}_0\int\limits^{2}_{1}xye^{xy^2} dxdy &= \int\limits^{2}_{1}\int\limits^{1}_0xye^{xy^2} dydx \\
               &\text{SUBSTITUTION MIT $xy^2$. ABER WARUM WERDEN DIE GRENZEN NICHT ANGEPASST?}\\
               &= \frac 12 \left. \int\limits^{2}_{1} e^{xy^2} \right|^1_0 dx = \frac 12 \int\limits^{2}_{1} (e^x - 1) dx \\
               &= \frac 12 (e^x - x)|^2_1 = \frac 12 (e^2 - 2 - (e - 1)) = \frac 12 (e^2 - e - 1)
            \end{align*}
   \item 
            \begin{align*}
               \int\limits^{2}_{1}\int\limits^1_0\int\limits^y_0(x+1)z^x dzdydx &= \int\limits^{2}_{1}\int\limits^1_0 \left. z^{x+1} \right|^y_0 dydx \\
               &= \int\limits^{2}_{1}\int\limits^1_0 y^{x+1} dydx \\
               &= \int\limits^{2}_{1}\left. \frac 1 {x+2} y^{x+2} \right|^1_0 dx \\
               &= \int\limits^{2}_{1}\frac 1 {x+2} dx \\
               &= \left.\log(x+2)\right|^2_1 = \log(4) - \log(3) = \log\left(\frac 4 3\right)
            \end{align*}
   \item 
            \begin{align*}
               \int\limits^{1}_{0}\int\limits^{1+y}_{\sqrt{y}}(xy^3) dxdy &=   \int\limits^{1}_{0} \left. (\frac 1 2 x^2y^3) \right|^{1+y}_{\sqrt{y}} dy \\
               &=   \int\limits^{1}_{0} (\frac 1 2 (y^5 + y^4 + y^3))  dy = \frac 1 2 \int\limits^{1}_{0} y^5 + y^4 + y^3  dy \\
               &= \left.\frac 1 2\left( \frac 16 y^6 + \frac 15 y^5 + \frac 14 y^3 \right)\right|^1_0 = \frac 12\left( \frac 16 + \frac 15 + \frac 14 \right) = \frac{37}{120}
            \end{align*}
\end{enumerate}
\section*{Aufgabe 2}
\begin{align*}
\mathcal{T} &:= \left\lbrace (x,y,z)^T \in \R^3 \mid 0\leq x \leq 1, \; 0\leq y \text{?? TODO}\ \right\rbrace
\end{align*}
\section*{Aufgabe 3}
\begin{align*}
x &= r \sin( \vartheta ) \cos(\varphi) \\
y &= r \sin(\vartheta) \sin(\varphi)\\
z &= r \cos( \vartheta)\\
\text{und der Funktionaldeterminanten} \\
dV &= r^2 \sin \vartheta~ dr ~d\vartheta ~d\varphi \\
\iiint\limits_{\mathcal{K}_R} \left( x^2 + y^2 \right) dV &= \int\limits_{0}^{R}\int\limits_{0}^{\pi} \int\limits_{0}^{2\pi} \left( r^2\sin^2(\vartheta)\cos^2(\varphi) + r^2\sin^2(\vartheta)\sin^2(\varphi) \right) r^2 \sin(\vartheta)~ dr \; d\varphi \; d\vartheta\\ 
&= \int\limits_{0}^{\pi} \int\limits_{0}^{2\pi}\int\limits_{0}^{R} \left( r^2\sin^2(\vartheta) \left( \cos^2(\varphi) + \sin^2(\varphi) \right) \right)  r^2 \sin(\vartheta)~ dr ~d\vartheta ~d\varphi\\
&= \int\limits_{0}^{\pi} \int\limits_{0}^{2\pi}\int\limits_{0}^{R}  r^2\sin^2(\vartheta)  r^2 \sin \vartheta~ dr ~d\vartheta ~d\varphi\\
&= \int\limits_{0}^{\pi} \int\limits_{0}^{2\pi}\int\limits_{0}^{R}  r^4\sin^3(\vartheta) dr \; d\varphi \; d\vartheta\\
&= \int\limits_{0}^{\pi} \sin^3(\vartheta) \int\limits_{0}^{2\pi}\int\limits_{0}^{R} r^4 dr \; d\varphi \; d\vartheta\\
&= \int\limits_{0}^{\pi} \sin^3(\vartheta)  \int\limits_{0}^{2\pi} \frac 15 R^5 d\varphi \; d\vartheta\\
&= \int\limits_{0}^{\pi} \sin^3(\vartheta)\frac 15 R^5 2\pi \; d\vartheta\\
&= \frac 25 \pi R^5 \int\limits_0^{\pi} \sin^3(\vartheta) \; d\vartheta\\
&= \frac 25 \pi R^5 \cdot \frac 43
\end{align*}
\section*{Aufgabe 4}
Wir kippen nicht das Glas sondern definieren, das die Gravitationskraft nicht mehr in $-\vec e_y$ sondern in $-\sqrt 2 (\vec e_x + \vec e_y)$-Richtung zeigt. Das heißt, wenn das Glas um $45^{\circ}$ gekippt wird, verläuft die Oberfläche des Bieres entlang $y_B = x + 2$, da die Oberfläche einer Flüssigkeit in Ruhe stets orthogonal zum Gravitationsvektor ist. Die Verschiebung in $\vec e_y$-Richtung kommt daher, dass das Glas voll gefüllt ist und das Bier die Kante des Bierglases berührt und demzufolge durch den Punkt $(2,4)$ geht. Um die Integrationsgrenzen zu erhalten, berechnen wir die Schnittstellen beider Graphen:
\begin{align*}
x^2 &= x + 2 \Rightarrow x_{1/2} =  \frac 12 \pm \sqrt{ \frac 14 + 2} = \frac 12 \pm 1,5\\
\Rightarrow &x_1 = -1\\
\Rightarrow &x_2 = 2
\end{align*}
Die Menge an "'zweidimensionalem Bier"' berechnet sich nun wie folgt:
\begin{align*}
A_B &= \int\limits_{-1}^{2} (x+2) - x^2 dx\\
&= -\frac 13 x^3 + \frac 12 x^2 + 2x \mid_{-1}^{2}\\
&= 2\left(-\frac 13 2^2 + \frac 12 2 + 2\right) + \left(-\frac 13 -\frac 12 +2\right)\\
&= -\frac 83 + 6 - \frac 56 + 2 = 8 - \frac {21}6 = \frac {27}6
\end{align*}
\end{document}


